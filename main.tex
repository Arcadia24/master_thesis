% !TeX root = main.tex
%%%%%%%%%%%%%%%%%%%%%%%%%%%%%%%%%%%%%%%%%
% kaobook
% LaTeX Template
% Version 1.2 (4/1/2020)
%
% This template originates from:
% https://www.LaTeXTemplates.com
%
% For the latest template development version and to make contributions:
% https://github.com/fmarotta/kaobook
%
% Authors:
% Federico Marotta (federicomarotta@mail.com)
% Based on the doctoral thesis of Ken Arroyo Ohori (https://3d.bk.tudelft.nl/ken/en)
% and on the Tufte-LaTeX class.
% Modified for LaTeX Templates by Vel (vel@latextemplates.com)
%
% License:
% CC0 1.0 Universal (see included MANIFEST.md file)
%
%%%%%%%%%%%%%%%%%%%%%%%%%%%%%%%%%%%%%%%%%

%----------------------------------------------------------------------------------------
%	PACKAGES AND OTHER DOCUMENT CONFIGURATIONS
%----------------------------------------------------------------------------------------

\documentclass[
	fontsize=10pt, % Base font size
	twoside=false, % Use different layouts for even and odd pages (in particular, if twoside=true, the margin column will be always on the outside)
	%open=any, % If twoside=true, uncomment this to force new chapters to start on any page, not only on right (odd) pages
	%chapterprefix=true, % Uncomment to use the word "Chapter" before chapter numbers everywhere they appear
	%chapterentrydots=true, % Uncomment to output dots from the chapter name to the page number in the table of contents
	numbers=noenddot, % Comment to output dots after chapter numbers; the most common values for this option are: enddot, noenddot and auto (see the KOMAScript documentation for an in-depth explanation)
	%draft=true, % If uncommented, rulers will be added in the header and footer
	%overfullrule=true, % If uncommented, overly long lines will be marked by a black box; useful for correcting spacing problems
]{kaobook}
%!TEX root = ../thesis.tex

%\hypersetup{colorlinks,linktocpage,urlcolor=red}
%
%\definecolor{myGreen}{HTML}{05C18E} 
%\definecolor{myGreenDarker}{HTML}{178C6C} \colorlet{mylinkcolor}{green!50!black}

\definecolor{webbrown}{rgb}{.6,0,0}

\hypersetup{
  colorlinks=true,
  linkcolor=black, %myGreenDarker
%  urlcolor=myGreenDarker,
  citecolor = webbrown,
  urlcolor=webbrown,
  hyperfootnotes=false,
  hypertexnames,
  bookmarks=true}
  
%\setsidenotefont{\color{black}\footnotesize}   <-- set the color and font here
%\setmarginnotefont{\color{black}\footnotesize} <-- and here
%
%\renewcommand{\maketitlepage}[0]{%
%  \cleardoublepage%
%  {%
%  \sffamily%
%  \begin{fullwidth}%
%  \fontsize{18}{20}\selectfont\par\noindent\textcolor{darkgray}{\allcaps{\thanklessauthor}}%
%  \vspace{11.5pc}%
%  \fontsize{24}{45}\selectfont\par\noindent\textcolor{darkgray}{\allcaps{\thanklesstitle}}
%  \fontsize{17.4}{25}\selectfont\par\noindent\textcolor{darkgray}{\allcaps{For Affective Touch Communication}}%
%  \fontsize{10.0}{17}\selectfont\par\noindent\textcolor{Gray}{\allcaps{Devices that touch to convey emotions and feel that contact}}%
%
%  \vfill%
%  \fontsize{14}{16}\selectfont\par\noindent\allcaps{\thanklesspublisher}%
%  \end{fullwidth}%
%  }
 %  \thispagestyle{empty}%
%  \clearpage%
%}

%%%% Kevin Godny's code for title page and contents from https://groups.google.com/forum/#!topic/tufte-latex/ujdzrktC1BQ
% \makeatletter
% \renewcommand{\maketitlepage}{%
% \begingroup%
% \setlength{\parindent}{0pt}

% {\fontsize{24}{24}\selectfont\textit{\@author}\par}

% \vspace{1.75in}{\fontsize{36}{54}\selectfont\@title\par}

% % \vspace{0.5in}{\fontsize{14}{14}\selectfont\textsf{\smallcaps{\@date}}\par}
% \vspace{0.5in}{\fontsize{14}{14}\selectfont\textsf{\sc\@date}\par}

% \vfill{\fontsize{14}{14}\selectfont\textit{\@publisher}\par}

% \thispagestyle{empty}
% \endgroup
% }
% \makeatother

%\titlecontents{part}%
%    [0pt]% distance from left margin
%    {\addvspace{0.25\baselineskip}}% above (global formatting of entry)
%    {\allcaps{Part~\thecontentslabel}\allcaps}% before w/ label (label = ``Part I'')
%    {\allcaps{Part~\thecontentslabel}\allcaps}% before w/o label
%    {}% filler and page (leaders and page num)
%    [\vspace*{0.5\baselineskip}]% after
%
%
%\titlecontents{chapter}%
%    [4em]% distance from left margin
%    {}% above (global formatting of entry)
%    {\contentslabel{2em}\textit}% before w/ label (label = ``Chapter 1'')
%    {\hspace{0em}\textit}% before w/o label
%    {\qquad\thecontentspage}% filler and page (leaders and page num)
%    [\vspace*{0.5\baselineskip}]% after
%%%%% End additional code by Kevin Godby


%% CHANGE CITE COMMAND
\renewcommand{\cite}[1]{%
~\citep{#1}%
}



%%
% If they're installed, use Bergamo and Chantilly from www.fontsite.com.
% They're clones of Bembo and Gill Sans, respectively.
%\IfFileExists{bergamo.sty}{\usepackage[osf]{bergamo}}{}% Bembo
%\IfFileExists{chantill.sty}{\usepackage{chantill}}{}% Gill Sans

%%%%%%%%%%%%%%%%%%%%%%%%%%%%%%%%%%%%%%%%%%%%%%%%%%%%%%%%%%
%%% INCLUSION / EXCLUSION %%%%%%%%%%%%%%%%%%%
\usepackage{microtype}
\usepackage{comment}
% !!! Comment or uncomment line under to exclude or include the content of the chapter:
%\excludecomment{content} % exclude the content, (only get introduction and summary)
\includecomment{content} % include the content, (get eveevolutionrything)
\includecomment{export}
%%%%%%%%%%%%%%%%%%%%%%%%%%%%%%%%%%%%%%%%%%%%%%%%%%%%%%%%%%


%%
% For nicely typeset tabular material
\usepackage{booktabs}
%%
% For graphics / images
\usepackage{graphicx}
\setkeys{Gin}{width=\linewidth,totalheight=\textheight,keepaspectratio}
\graphicspath{{graphics/}}
% The fancyvrb package lets us customize the formatting of verbatim environments.  We use a slightly smaller font.
\usepackage{fancyvrb}
\fvset{fontsize=\normalsize}

%%
% Prints argument within hanging parentheses (i.e., parentheses that take
% up no horizontal space).  Useful in tabular environments.
% \newcommand{\hangp}[1]{\makebox[0pt][r]{(}#1\makebox[0pt][l]{)}}

%%
% Prints an asterisk that takes up no horizontal space.
% Useful in tabular environments.
% \newcommand{\hangstar}{\makebox[0pt][l]{*}}

%%
% Prints a trailing space in a smart way.
\usepackage{xspace}

%
%%%
%% Some shortcuts for Tufte's book titles.  The lowercase commands will
%% produce the initials of the book title in italics.  The all-caps commands
%% will print out the full title of the book in italics.
%\newcommand{\vdqi}{\textit{VDQI}\xspace}
%\newcommand{\ei}{\textit{EI}\xspace}
%\newcommand{\ve}{\textit{VE}\xspace}
%\newcommand{\be}{\textit{BE}\xspace}
%%\newcommand{\VDQI}{\textit{Visualizing dynamic social data  with rationally designed constructive systems}\xspace}
%\newcommand{\EI}{\textit{Envisioning Information}\xspace}
%\newcommand{\VE}{\textit{Visual Explanations}\xspace}
%\newcommand{\BE}{\textit{Beautiful Evidence}\xspace}
%\newcommand{\TL}{Tufte-\LaTeX\xspace}

% Prints the month name (e.g., January) and the year (e.g., 2008)
% \newcommand{\monthyear}{%
%   \ifcase\month\or January\or February\or March\or April\or May\or June\or
%   July\or August\or September\or October\or November\or
%   December\fi\space\number\year
% }





% Prints an epigraph and speaker in sans serif, all-caps type.
\newcommand{\openepigraph}[2]{%
  %\sffamily\fontsize{14}{16}\selectfont
  \begin{fullwidth}
  \sffamily\large
  \begin{doublespace}
  \noindent\allcaps{#1}\\% epigraph
  \noindent\allcaps{#2}% author
  \end{doublespace}
  \end{fullwidth}
}

% Inserts a blank page
% \newcommand{\blankpage}{\newpage\hbox{}\thispagestyle{empty}\newpage}

\usepackage{units}

% Typesets the font size, leading, and measure in the form of 10/12x26 pc.
\newcommand{\measure}[3]{#1/#2$\times$\unit[#3]{pc}}

% Macros for typesetting the documentation
\newcommand{\hlred}[1]{\textcolor{Green}{#1}}% prints in red
\newcommand{\hangleft}[1]{\makebox[0pt][r]{#1}}
% \newcommand{\hairsp}{\hspace{1pt}}% hair space
\newcommand{\hquad}{\hskip0.5em\relax}% half quad space
\newcommand{\TODO}{\textcolor{red}{\bf TODO!}\xspace}
% \newcommand{\ie}{\textit{i.\hairsp{}e.}\xspace}
% \newcommand{\eg}{\textit{e.\hairsp{}g.}\xspace}
% \newcommand{\na}{\quad--}% used in tables for N/A cells
\providecommand{\XeLaTeX}{X\lower.5ex\hbox{\kern-0.15em\reflectbox{E}}\kern-0.1em\LaTeX}
\newcommand{\tXeLaTeX}{\XeLaTeX\index{XeLaTeX@\protect\XeLaTeX}}
% \index{\texttt{\textbackslash xyz}@\hangleft{\texttt{\textbackslash}}\texttt{xyz}}
\newcommand{\tuftebs}{\symbol{'134}}% a backslash in tt type in OT1/T1
\newcommand{\doccmdnoindex}[2][]{\texttt{\tuftebs#2}}% command name -- adds backslash automatically (and doesn't add cmd to the index)
\newcommand{\doccmddef}[2][]{%
  \hlred{\texttt{\tuftebs#2}}\label{cmd:#2}%
  \ifthenelse{\isempty{#1}}%
    {% add the command to the index
      \index{#2 command@\protect\hangleft{\texttt{\tuftebs}}\texttt{#2}}% command name
    }%
    {% add the command and package to the index
      \index{#2 command@\protect\hangleft{\texttt{\tuftebs}}\texttt{#2} (\texttt{#1} package)}% command name
      \index{#1 package@\texttt{#1} package}\index{packages!#1@\texttt{#1}}% package name
    }%
}% command name -- adds backslash automatically
\newcommand{\doccmd}[2][]{%
  \texttt{\tuftebs#2}%
  \ifthenelse{\isempty{#1}}%
    {% add the command to the index
      \index{#2 command@\protect\hangleft{\texttt{\tuftebs}}\texttt{#2}}% command name
    }%
    {% add the command and package to the index
      \index{#2 command@\protect\hangleft{\texttt{\tuftebs}}\texttt{#2} (\texttt{#1} package)}% command name
      \index{#1 package@\texttt{#1} package}\index{packages!#1@\texttt{#1}}% package name
    }%
}% command name -- adds backslash automatically
\newcommand{\docopt}[1]{\ensuremath{\langle}\textrm{\textit{#1}}\ensuremath{\rangle}}% optional command argument
\newcommand{\docarg}[1]{\textrm{\textit{#1}}}% (required) command argument
\newenvironment{docspec}{\begin{quotation}\ttfamily\parskip0pt\parindent0pt\ignorespaces}{\end{quotation}}% command specification environment
\newcommand{\docenv}[1]{\texttt{#1}\index{#1 environment@\texttt{#1} environment}\index{environments!#1@\texttt{#1}}}% environment name
\newcommand{\docenvdef}[1]{\hlred{\texttt{#1}}\label{env:#1}\index{#1 environment@\texttt{#1} environment}\index{environments!#1@\texttt{#1}}}% environment name
\newcommand{\docpkg}[1]{\texttt{#1}\index{#1 package@\texttt{#1} package}\index{packages!#1@\texttt{#1}}}% package name
\newcommand{\doccls}[1]{\texttt{#1}}% document class name
\newcommand{\docclsopt}[1]{\texttt{#1}\index{#1 class option@\texttt{#1} class option}\index{class options!#1@\texttt{#1}}}% document class option name
\newcommand{\docclsoptdef}[1]{\hlred{\texttt{#1}}\label{clsopt:#1}\index{#1 class option@\texttt{#1} class option}\index{class options!#1@\texttt{#1}}}% document class option name defined
\newcommand{\docmsg}[2]{\bigskip\begin{fullwidth}\noindent\ttfamily#1\end{fullwidth}\medskip\par\noindent#2}
\newcommand{\docfilehook}[2]{\texttt{#1}\index{file hooks!#2}\index{#1@\texttt{#1}}}
\newcommand{\doccounter}[1]{\texttt{#1}\index{#1 counter@\texttt{#1} counter}}




%\geometry{textwidth=.55\paperwidth}


% Generates the index
\usepackage{makeidx}
\makeindex



% Nomenclature
%\usepackage{nomencl}
%\renewcommand{\nomname}{List of Abbreviations}
%\makenomenclature



%%%%
\makeatletter
\renewcommand*\l@figure{\@dottedtocline{1}{1.5em}{2.3em}}
\makeatother

%% change TOC
\setcounter{tocdepth}{2}
\setcounter{secnumdepth}{2}

%%%%%%%%%%%%%%%%%%%%%%%%%%%%%%%%%%%%%%%%%%%%%%%%%%
%%%%%%%%%%%%%%%%%%%%%%%%%%%%%%%%%%%%%%%%%%%%%%%%%%
\usepackage{amssymb}% http://ctan.org/pkg/amssymb
\usepackage{pifont}% http://ctan.org/pkg/pifont
%\usepackage{graphics} % for EPS, load graphicx instead
\usepackage{graphicx}
\usepackage{multirow}
\usepackage{xspace}
\usepackage{tabularx}
\usepackage{color}
\usepackage{listings}
\usepackage[normalem]{ulem}
\usepackage{colortbl}
\usepackage{morefloats}
\usepackage{enumitem}
\usepackage{rotating}
\usepackage{comment}
\usepackage{rotating}
% \usepackage[sort, numbers]{natbib} 
\usepackage[retainorgcmds]{IEEEtrantools}
\usepackage{bibentry}
\usepackage{longtable}
\usepackage{glossaries}
\usepackage{gensymb}
\usepackage{csvsimple}
\usepackage{amsmath}
\usepackage{cleveref}% Has to be loaded after hyperref
\usepackage[utf8]{inputenc}
\usepackage{todonotes}
\usepackage{marginfix}
\usepackage[export]{adjustbox}
\usepackage{fullwidth}
%\setkeys{Gin}{height=2cm}
%\usepackage{float}
\usepackage[caption=false]{subfig}

\usepackage[strict]{changepage}

\setlist[description]{style = multiline, labelwidth = 55pt}
\usepackage[parfill]{parskip}
\makeatletter
% Paragraph indentation and separation for normal text
% \renewcommand{\@tufte@reset@par}{%
%   \setlength{\RaggedRightParindent}{1.0pc}%
%   \setlength{\parindent}{1pc}%
%   \setlength{\parskip}{8pt}%\baselineskip % default 12pt for 10pt font
% }
% \@tufte@reset@par

% Paragraph indentation and separation for marginal text
% \renewcommand{\@tufte@margin@par}{%
%   \setlength{\RaggedRightParindent}{0.5pc}%
%   \setlength{\JustifyingParindent}{0.0pc}%
%   \setlength{\parindent}{0.5pc}%
%   \setlength{\parskip}{6pt}%
% }
\makeatother





%% Correction

%\newcommand{\Ssubsection}[1]{{\setlength{\parindent}{0cm}\normalfont{\textit{\newline#1}}}\newline}
%\newcommand{\Ssubsection}[1]{{\setlength{\parindent}{0cm}\normalfont{\textit{#1}}}}




\usepackage{mdframed}
\newmdenv[
  leftmargin = 0pt,
  innerleftmargin = 1em,
  innerrightmargin = 0pt,
 innerbottommargin = 0pt,
  innertopmargin = 0pt,
  rightmargin = 0pt,
  linewidth = 2pt,
  topline = false,
  rightline = false,
  bottomline = false,
  skipabove = 6pt
  ]{leftbar}


\newcommand{\mframe}[1]{\begin{leftbar}{#1}\end{leftbar}}


%You can copy those commands to the preamble of your document and fill in the values that you prefer (e.g., 0pt for the indents and \baselineskip for the \parskip).








% \titleclass{\subsubsection}{straight}
% \titleformat{\subsubsection}%
%   [hang]% shape
%   {\normalfont\large\itshape}% format applied to label+text
%   {\thesubsubsection}% label
%   {1em}% horizontal separation between label and title body
%   {}% before the title body
%   []% after the title body
  
  
  

%%%%%%%%%%%%%%%%%%%%%%%%%%%%%%%%%%%%%
%%%%%% FANCY FRAMES
%% https://tex.stackexchange.com/questions/348501/example-of-box-inside-box
%%%%%%%%%%%%%%%%%%%%%%%%
%\usepackage[margin=0.5in]{geometry}
%\usepackage{tikz,lipsum,lmodern}
\usepackage{tikz,lipsum}
\usepackage[most]{tcolorbox}

\tcbset{titre/.style={boxed title style={boxrule=0pt,colframe=white}}}

\definecolor{gradientGreenL}{HTML}{1fe2ad} 
\definecolor{gradientGreenR}{HTML}{d4eb6f} 


\newtcolorbox{BoxResume}[2][]{
                boxrule=0.5pt,
                colback=white,
                top=3pt,bottom=2pt,left=2pt,right=2pt,
                colframe=webbrown,
                fonttitle=\sffamily\small,%\bfseries
                coltitle=black,
                colbacktitle=white,
                enhanced,
                attach boxed title to top left={xshift=5mm, yshift=-2mm},
                title=#2,#1
                }


\newtcolorbox{BoxIn}{
enhanced,
colframe=white,
interior style={
left color=gradientGreenL!7!white,
right color=gradientGreenR!7!white},
%frame style image=background\aa.jpg
left=5mm,
top=4mm,
bottom=4mm,
right=5mm,
boxsep=0mm,
nobeforeafter}



\newtcolorbox{BoxResumeNew}[2][]{
                boxrule=1pt,
                colback=white,
                top=3pt,bottom=2pt,left=2pt,right=2pt,
                colframe=black,
                fonttitle=\sffamily\small,%\bfseries
                coltitle=black,
                colbacktitle=white,
                enhanced,
                attach boxed title to top left={xshift=5mm, yshift=-2mm},
                title=#2,#1
                }


\newtcolorbox{BoxInNew}{
enhanced,
colframe=white,
colback=black!2!white,
%frame style image=background\aa.jpg
left=5mm,
top=4mm,
bottom=4mm,
right=5mm,
boxsep=0mm,
nobeforeafter}



\newcommand{\remember}[1]{
\vspace*{\fill}
\begin{BoxResumeNew}[titre]{WHAT YOU MUST REMEMBER}
 \begin{BoxInNew}{}
 #1
 \end{BoxInNew}{}
\end{BoxResumeNew}
\vspace{0.5cm}
} 

%%%% USAGE

%\remember{
%\textit{Contributions:}\vspace{0.5em}
%\begin{itemize}
%\item[$-$] Design and development of a finger robotic actuator for mobile devices
%\item[$-$] Applications and scenarios that demonstrate its use as a medium,  as a tool and as a virtual partner
%\item[$-$] Initial evaluation of perception of the appearance and the relevance of scenarios
%\end{itemize}
%}


%%%%%%%%%%%%%%%%%%%%%%%%%%%%%%
%%%%%%%%%   QUOTE %%%%%%%%%%%%%%%%%%%
%%%%%%%%%%%%%%%%%%

\makeatletter
\renewcommand{\@chapapp}{}% Not necessary...
\newenvironment{chapquote}[2][2em]
  {\setlength{\@tempdima}{#1}%
   \def\chapquote@author{#2}%
   \parshape 1 \@tempdima \dimexpr\textwidth-2\@tempdima\relax%
   }
  {\par\normalfont\hfill--\ \chapquote@author\hspace*{\@tempdima}\par\bigskip}
\makeatother


%\listfiles

% Set the language
\usepackage[english]{babel} % Load characters and hyphenation
\usepackage[english=british]{csquotes} % English quotes

% Load packages for testing
\usepackage{blindtext}
%\usepackage{showframe} % Uncomment to show boxes around the text area, margin, header and footer
%\usepackage{showlabels} % Uncomment to output the content of \label commands to the document where they are used


\lstset{
    language=Python, % Language
    basicstyle=\ttfamily\footnotesize, % Font style
    keywordstyle=\color{blue}\ttfamily,
    stringstyle=\color{red}\ttfamily,
    commentstyle=\color{gray}\ttfamily\itshape,
    showstringspaces=false,
    breaklines=true, % Automatic line breaking
    tabsize=4, % Tab size
    numbers=left, % Line numbers
    numberstyle=\tiny\color{gray}, % Line number style
    stepnumber=1, % Line number increment
    numbersep=5pt, % Line number separation
    frame=single, % Frame around code
    framexleftmargin=5mm, % Frame margin
    captionpos=b, % Caption position
    xleftmargin=2em, % <-- Adjust horizontal spacing here
    escapeinside={\%*}{*)} % If you want to add LaTeX within your code
}

% Load the bibliography package
\usepackage{styles/kaobiblio}
\addbibresource{main.bib} % Bibliography file

% Load mathematical packages for theorems and related environments. NOTE: choose only one between 'mdftheorems' and 'plaintheorems'.
\usepackage{styles/mdftheorems}
%\usepackage{styles/plaintheorems}

\graphicspath{{examples/documentation/images/}{images/}} % Paths in which to look for images

\makeindex[columns=3, title=Alphabetical Index, intoc] % Make LaTeX produce the files required to compile the index

\makeglossaries % Make LaTeX produce the files required to compile the glossary

\makenomenclature % Make LaTeX produce the files required to compile the nomenclature

% Reset sidenote counter at chapters
%\counterwithin*{sidenote}{chapter}
% \setcounter{section}{-1}
%----------------------------------------------------------------------------------------

\newcommand{\red}[1]{\textcolor[rgb]{1, 0, 0}{#1}}
\newcommand{\problematic}{How can sensing techniques redefine our interaction with plants ?}
%% This section is for removing the 0.*** in the sections
\makeatletter
\renewcommand{\thesection}{%
  \ifnum\c@chapter<1 \@arabic\c@section
  \else \thechapter.\@arabic\c@section
  \fi
}
\makeatother
%%%%%%%%%%%%%%%%%%%%%%%%

\begin{document}
\def\title#1{\gdef\@title{#1}\gdef\THETITLE{#1}}
%----------------------------------------------------------------------------------------
%	BOOK INFORMATION
%----------------------------------------------------------------------------------------

% \titlehead{The \texttt{kaobook} class}
\subject{Master thesis}
\title{\problematic}
\subtitle{\problematic}

\author{Nathan VIDAL}

\date{\today}

\titlehead{\centering\includegraphics[width=6cm]{images/ift_logo.png}}

\publishers{\textbf{Supervisor}\\ Xiao Xiao}


%----------------------------------------------------------------------------------------

\frontmatter % Denotes the start of the pre-document content, uses roman numerals

%----------------------------------------------------------------------------------------
%	OPENING PAGE
%----------------------------------------------------------------------------------------

% \makeatletter
% \extratitle{
% 	% In the title page, the title is vspaced by 9.5\baselineskip
% 	\vspace*{9\baselineskip}
% 	\vspace*{\parskip}
% 	\begin{center}
% 		% In the title page, \huge is set after the komafont for title
% 		\usekomafont{title}\huge\@title
% 	\end{center}
% }
% \makeatother

%----------------------------------------------------------------------------------------
%	COPYRIGHT PAGE
%----------------------------------------------------------------------------------------

% \makeatletter
% \uppertitleback{\@titlehead} % Header

% \lowertitleback{
% 	\textbf{Disclaimer}\\
% 	You can edit this page to suit your needs. For instance, here we have a no copyright statement, a colophon and some other information. This page is based on the corresponding page of Ken Arroyo Ohori's thesis, with minimal changes.

% 	\medskip

% 	\textbf{No copyright}\\
% 	\cczero\ This book is released into the public domain using the CC0 code. To the extent possible under law, I waive all copyright and related or neighbouring rights to this work.

% 	To view a copy of the CC0 code, visit: \\\url{http://creativecommons.org/publicdomain/zero/1.0/}

% 	\medskip

% 	\textbf{Colophon} \\
% 	This document was typeset with the help of \href{https://sourceforge.net/projects/koma-script/}{\KOMAScript} and \href{https://www.latex-project.org/}{\LaTeX} using the \href{https://github.com/fmarotta/kaobook/}{kaobook} class.

% 	The source code of this book is available at:\\\url{https://github.com/fmarotta/kaobook}

% 	(You are welcome to contribute!)

% 	\medskip

% 	\textbf{Publisher} \\
% 	First printed in May 2019 by \@publishers
% }
% \makeatother

%----------------------------------------------------------------------------------------
%	DEDICATION
%----------------------------------------------------------------------------------------

% \dedication{
% 	The harmony of the world is made manifest in Form and Number, and the heart and soul and all the poetry of Natural Philosophy are embodied in the concept of mathematical beauty.\\
% 	\flushright -- D'Arcy Wentworth Thompson
% }

%----------------------------------------------------------------------------------------
%	OUTPUT TITLE PAGE AND PREVIOUS
%----------------------------------------------------------------------------------------

% Note that \maketitle outputs the pages before here

% If twoside=false, \uppertitleback and \lowertitleback are not printed
% To overcome this issue, we set twoside=semi just before printing the title pages, and set it back to false just after the title pages
\KOMAoptions{twoside=semi}
\maketitle
\KOMAoptions{twoside=false}



%----------------------------------------------------------------------------------------
%	TABLE OF CONTENTS & LIST OF FIGURES/TABLES
%----------------------------------------------------------------------------------------

\begingroup % Local scope for the following commands

% Define the style for the TOC, LOF, and LOT
%\setstretch{1} % Uncomment to modify line spacing in the ToC
% \hypersetup{linkcolor=blue} % Uncomment to set the colour of links in the ToC
\setlength{\textheight}{23cm} % Manually adjust the height of the ToC pages

% Turn on compatibility mode for the etoc package
\etocstandarddisplaystyle % "toc display" as if etoc was not loaded
\etocstandardlines % toc lines as if etoc was not loaded

% Comment both of the following lines to have the LOF and the LOT on different pages
\let\cleardoublepage\bigskip
\let\clearpage\bigskip

\setcounter{tocdepth}{3}
\setcounter{secnumdepth}{3}

% \listoftables % Output the list of tables

\tableofcontents



\endgroup

%----------------------------------------------------------------------------------------
%	MAIN BODY
%----------------------------------------------------------------------------------------

\mainmatter % Denotes the start of the main document content, resets page numbering and uses arabic numbers
\setchapterstyle{kao} % Choose the default chapter heading style

\vspace*{\fill}

\section*{Abstract}
\textit{This thesis explores the integration of sound-based artificial intelligence and augmented reality to develop interactive systems for environmental monitoring, emotional expression, and immersive learning. The first project uses machine learning to classify bird species through vocalizations, contributing to accessible bioacoustic monitoring. The second project develops an emotionally adaptive text-to-speech (TTS) system, enhancing Human-Computer Interaction with lifelike emotional expressions. The final project combines TTS and AR in a theater training platform, enabling students to interact with virtual characters and receive real-time feedback. These projects demonstrate the potential of AI and AR to transform Human-Computer Interaction in conservation, education, and performance arts..}
\vspace*{\fill}

\section{Introduction}

\subsection{Background motivation}

I am an creative technology engineer that is passionate about embedded systems and 
their hardware/software architecture.
Pushed by my principal investigator and motivated by challenges, I wanted to 
explore the intersection of biology and electronics. 
I aim to transform plants into bio-sensors, using their natural sensing capabilities to 
capture the human-plant interaction. Extending the capacities of a single plant, I want to create a network of 
plant-based sensors. 

I am also interested in the use of sensor data. With no particular appetence for musical creation, 
my principal investigator challenged me onto create a device that can use the data from the plant 
and generate sound based on interaction. The musical generation allows the plant to be listened to and to \hl{care
about it.}

\subsection{Context and overview}

This research is in line with the new means of interaction and new sensors that surround us every day.
This master's thesis seeks to use the natural capacities of plants, which are made up of thousands of sensors, and to understand them.
This could make it possible to create plant networks and monitor the state of our green partners. 
At the same time, it could reduce the amount of silicon needed to deploy a sensor field.
It could also open up new possibilities in the field of Human Computer Interface research by adding a new interface.


\subsection{Problematic}

The main problematic that this master thesis will focus on is :

\begin{center}
    \textbf{How can sensing technologies redefine our interactions with plants ?}\\
\end{center}

\subsection{Research domain}

Research domains on the human-plant interaction are wide. The HCI\footnote{Human Computer Interaction} field is focused
during this master thesis. The Human Computer Interaction field focuses on the interfaces between people and computers.
This field is at the intersection "between psychology and social sciences, on the one hand, and computer science and technology,
on the other" \cite{carrollHUMANCOMPUTERINTERACTIONPsychology}. This master thesis aims to work onto the interaction
we have with plants and nature and to enhance plant capabilities.

The plant is transformed is used as a living sensor and thus the project is reaching the instrumentation engineering 
and electronic field. This field aims to think and create new way of capturing data to make sensors.
A bio-living sensor such as the plant needs to be understood using sensing techniques.

The sensor making and creation field is also focused. In this master thesis plants are transformed into sensor.
This transformation...


\subsection{Contributions}

\vspace*{\fill}

\section*{Acknowledgements}

I would like to express my heartfelt gratitude to the three principal investigators who have guided me over the past three years. My deepest thanks go to Marc Teyssier for his invaluable support and mentorship during my UROP year, as well as to Clément Duhart and Xiao Xiao for their guidance and assistance throughout these two years of my master’s program.

I am also deeply grateful to my classmates and friends for making these two years unforgettable. A special thanks to Hugo Devoille and Paul Even, as well as to Noé Guennoun, Nicolas Leboucher, Yohann Cossez, Marc-Adrien, and many others who have been part of this journey.

Lastly, I would like to thank my professors and researchers, including Yliess, Grégor, Thomas Juldo, Paul-Peter, and Madalina, for their unwavering support and the time they dedicated to teaching us. Their efforts have equipped us with the comprehensive knowledge and skills in engineering and research, preparing us to face future challenges with confidence.
\vspace*{\fill}





%----------------------------------------------------------------------------------------

\backmatter % Denotes the end of the main document content
\setchapterstyle{plain} % Output plain chapters from this point onwards

%----------------------------------------------------------------------------------------
%	BIBLIOGRAPHY
%----------------------------------------------------------------------------------------

% The bibliography needs to be compiled with biber using your LaTeX editor, or on the command line with 'biber main' from the template directory

\defbibnote{bibnote}{Here are the references in citation order.\par\bigskip} % Prepend this text to the bibliography
\printbibliography[heading=bibintoc, title=References, prenote=bibnote] % Add the bibliography heading to the ToC, set the title of the bibliography and output the bibliography note

%----------------------------------------------------------------------------------------
%	NOMENCLATURE
%----------------------------------------------------------------------------------------

% The nomenclature needs to be compiled on the command line with 'makeindex main.nlo -s nomencl.ist -o main.nls' from the template directory

% \nomenclature{$c$}{Speed of light in a vacuum inertial frame}
% \nomenclature{$h$}{Planck constant}

% \renewcommand{\nomname}{Notation} % Rename the default 'Nomenclature'
% \renewcommand{\nompreamble}{The next list describes several symbols that will be later used within the body of the document.} % Prepend this text to the nomenclature

% \printnomenclature % Output the nomenclature

%----------------------------------------------------------------------------------------
%	GREEK ALPHABET
% 	Originally from https://gitlab.com/jim.hefferon/linear-algebra
%----------------------------------------------------------------------------------------

% \vspace{1cm}

% {\usekomafont{chapter}Greek Letters with Pronounciation} \\[2ex]
% \begin{center}
% 	\newcommand{\pronounced}[1]{\hspace*{.2em}\small\textit{#1}}
% 	\begin{tabular}{l l @{\hspace*{3em}} l l}
% 		\toprule
% 		Character & Name & Character & Name \\ 
% 		\midrule
% 		$\alpha$ & alpha \pronounced{AL-fuh} & $\nu$ & nu \pronounced{NEW} \\
% 		$\beta$ & beta \pronounced{BAY-tuh} & $\xi$, $\Xi$ & xi \pronounced{KSIGH} \\ 
% 		$\gamma$, $\Gamma$ & gamma \pronounced{GAM-muh} & o & omicron \pronounced{OM-uh-CRON} \\
% 		$\delta$, $\Delta$ & delta \pronounced{DEL-tuh} & $\pi$, $\Pi$ & pi \pronounced{PIE} \\
% 		$\epsilon$ & epsilon \pronounced{EP-suh-lon} & $\rho$ & rho \pronounced{ROW} \\
% 		$\zeta$ & zeta \pronounced{ZAY-tuh} & $\sigma$, $\Sigma$ & sigma \pronounced{SIG-muh} \\
% 		$\eta$ & eta \pronounced{AY-tuh} & $\tau$ & tau \pronounced{TOW (as in cow)} \\
% 		$\theta$, $\Theta$ & theta \pronounced{THAY-tuh} & $\upsilon$, $\Upsilon$ & upsilon \pronounced{OOP-suh-LON} \\
% 		$\iota$ & iota \pronounced{eye-OH-tuh} & $\phi$, $\Phi$ & phi \pronounced{FEE, or FI (as in hi)} \\
% 		$\kappa$ & kappa \pronounced{KAP-uh} & $\chi$ & chi \pronounced{KI (as in hi)} \\
% 		$\lambda$, $\Lambda$ & lambda \pronounced{LAM-duh} & $\psi$, $\Psi$ & psi \pronounced{SIGH, or PSIGH} \\
% 		$\mu$ & mu \pronounced{MEW} & $\omega$, $\Omega$ & omega \pronounced{oh-MAY-guh} \\
% 		\bottomrule
% 	\end{tabular} \\[1.5ex]
% 	Capitals shown are the ones that differ from Roman capitals.
% \end{center}

%----------------------------------------------------------------------------------------
%	GLOSSARY
%----------------------------------------------------------------------------------------

% The glossary needs to be compiled on the command line with 'makeglossaries main' from the template directory

% \newglossaryentry{computer}{
% 	name=computer,
% 	description={is a programmable machine that receives input, stores and manipulates data, and provides output in a useful format}
% }

% Glossary entries (used in text with e.g. \acrfull{fpsLabel} or \acrshort{fpsLabel})
% \newacronym[longplural={Frames per Second}]{fpsLabel}{FPS}{Frame per Second}
% \newacronym[longplural={Tables of Contents}]{tocLabel}{TOC}{Table of Contents}

\setglossarystyle{listgroup} % Set the style of the glossary (see https://en.wikibooks.org/wiki/LaTeX/Glossary for a reference)
\printglossary[title=Special Terms, toctitle=List of Terms] % Output the glossary, 'title' is the chapter heading for the glossary, toctitle is the table of contents heading

%----------------------------------------------------------------------------------------
%	INDEX
%----------------------------------------------------------------------------------------

% The index needs to be compiled on the command line with 'makeindex main' from the template directory

\printindex % Output the index

%----------------------------------------------------------------------------------------
%	BACK COVER
%----------------------------------------------------------------------------------------

% If you have a PDF/image file that you want to use as a back cover, uncomment the following lines

%\clearpage
%\thispagestyle{empty}
%\null%
%\clearpage
%\includepdf{cover-back.pdf}

%----------------------------------------------------------------------------------------

\end{document}
