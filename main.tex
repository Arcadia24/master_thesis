% !TeX root = main.tex
%%%%%%%%%%%%%%%%%%%%%%%%%%%%%%%%%%%%%%%%%
% kaobook
% LaTeX Template
% Version 1.2 (4/1/2020)
%
% This template originates from:
% https://www.LaTeXTemplates.com
%
% For the latest template development version and to make contributions:
% https://github.com/fmarotta/kaobook
%
% Authors:
% Federico Marotta (federicomarotta@mail.com)
% Based on the doctoral thesis of Ken Arroyo Ohori (https://3d.bk.tudelft.nl/ken/en)
% and on the Tufte-LaTeX class.
% Modified for LaTeX Templates by Vel (vel@latextemplates.com)
%
% License:
% CC0 1.0 Universal (see included MANIFEST.md file)
%
%%%%%%%%%%%%%%%%%%%%%%%%%%%%%%%%%%%%%%%%%

%----------------------------------------------------------------------------------------
%	PACKAGES AND OTHER DOCUMENT CONFIGURATIONS
%----------------------------------------------------------------------------------------

\documentclass[
	fontsize=10pt, % Base font size
	twoside=false, % Use different layouts for even and odd pages (in particular, if twoside=true, the margin column will be always on the outside)
	%open=any, % If twoside=true, uncomment this to force new chapters to start on any page, not only on right (odd) pages
	%chapterprefix=true, % Uncomment to use the word "Chapter" before chapter numbers everywhere they appear
	%chapterentrydots=true, % Uncomment to output dots from the chapter name to the page number in the table of contents
	numbers=noenddot, % Comment to output dots after chapter numbers; the most common values for this option are: enddot, noenddot and auto (see the KOMAScript documentation for an in-depth explanation)
	%draft=true, % If uncommented, rulers will be added in the header and footer
	%overfullrule=true, % If uncommented, overly long lines will be marked by a black box; useful for correcting spacing problems
]{kaobook}
%!TEX root = ../thesis.tex

%\hypersetup{colorlinks,linktocpage,urlcolor=red}
%
%\definecolor{myGreen}{HTML}{05C18E} 
%\definecolor{myGreenDarker}{HTML}{178C6C} \colorlet{mylinkcolor}{green!50!black}

\definecolor{webbrown}{rgb}{.6,0,0}

\hypersetup{
  colorlinks=true,
  linkcolor=black, %myGreenDarker
%  urlcolor=myGreenDarker,
  citecolor = webbrown,
  urlcolor=webbrown,
  hyperfootnotes=false,
  hypertexnames,
  bookmarks=true}
  
%\setsidenotefont{\color{black}\footnotesize}   <-- set the color and font here
%\setmarginnotefont{\color{black}\footnotesize} <-- and here
%
%\renewcommand{\maketitlepage}[0]{%
%  \cleardoublepage%
%  {%
%  \sffamily%
%  \begin{fullwidth}%
%  \fontsize{18}{20}\selectfont\par\noindent\textcolor{darkgray}{\allcaps{\thanklessauthor}}%
%  \vspace{11.5pc}%
%  \fontsize{24}{45}\selectfont\par\noindent\textcolor{darkgray}{\allcaps{\thanklesstitle}}
%  \fontsize{17.4}{25}\selectfont\par\noindent\textcolor{darkgray}{\allcaps{For Affective Touch Communication}}%
%  \fontsize{10.0}{17}\selectfont\par\noindent\textcolor{Gray}{\allcaps{Devices that touch to convey emotions and feel that contact}}%
%
%  \vfill%
%  \fontsize{14}{16}\selectfont\par\noindent\allcaps{\thanklesspublisher}%
%  \end{fullwidth}%
%  }
 %  \thispagestyle{empty}%
%  \clearpage%
%}

%%%% Kevin Godny's code for title page and contents from https://groups.google.com/forum/#!topic/tufte-latex/ujdzrktC1BQ
% \makeatletter
% \renewcommand{\maketitlepage}{%
% \begingroup%
% \setlength{\parindent}{0pt}

% {\fontsize{24}{24}\selectfont\textit{\@author}\par}

% \vspace{1.75in}{\fontsize{36}{54}\selectfont\@title\par}

% % \vspace{0.5in}{\fontsize{14}{14}\selectfont\textsf{\smallcaps{\@date}}\par}
% \vspace{0.5in}{\fontsize{14}{14}\selectfont\textsf{\sc\@date}\par}

% \vfill{\fontsize{14}{14}\selectfont\textit{\@publisher}\par}

% \thispagestyle{empty}
% \endgroup
% }
% \makeatother

%\titlecontents{part}%
%    [0pt]% distance from left margin
%    {\addvspace{0.25\baselineskip}}% above (global formatting of entry)
%    {\allcaps{Part~\thecontentslabel}\allcaps}% before w/ label (label = ``Part I'')
%    {\allcaps{Part~\thecontentslabel}\allcaps}% before w/o label
%    {}% filler and page (leaders and page num)
%    [\vspace*{0.5\baselineskip}]% after
%
%
%\titlecontents{chapter}%
%    [4em]% distance from left margin
%    {}% above (global formatting of entry)
%    {\contentslabel{2em}\textit}% before w/ label (label = ``Chapter 1'')
%    {\hspace{0em}\textit}% before w/o label
%    {\qquad\thecontentspage}% filler and page (leaders and page num)
%    [\vspace*{0.5\baselineskip}]% after
%%%%% End additional code by Kevin Godby


%% CHANGE CITE COMMAND
\renewcommand{\cite}[1]{%
~\citep{#1}%
}



%%
% If they're installed, use Bergamo and Chantilly from www.fontsite.com.
% They're clones of Bembo and Gill Sans, respectively.
%\IfFileExists{bergamo.sty}{\usepackage[osf]{bergamo}}{}% Bembo
%\IfFileExists{chantill.sty}{\usepackage{chantill}}{}% Gill Sans

%%%%%%%%%%%%%%%%%%%%%%%%%%%%%%%%%%%%%%%%%%%%%%%%%%%%%%%%%%
%%% INCLUSION / EXCLUSION %%%%%%%%%%%%%%%%%%%
\usepackage{microtype}
\usepackage{comment}
% !!! Comment or uncomment line under to exclude or include the content of the chapter:
%\excludecomment{content} % exclude the content, (only get introduction and summary)
\includecomment{content} % include the content, (get eveevolutionrything)
\includecomment{export}
%%%%%%%%%%%%%%%%%%%%%%%%%%%%%%%%%%%%%%%%%%%%%%%%%%%%%%%%%%


%%
% For nicely typeset tabular material
\usepackage{booktabs}
%%
% For graphics / images
\usepackage{graphicx}
\setkeys{Gin}{width=\linewidth,totalheight=\textheight,keepaspectratio}
\graphicspath{{graphics/}}
% The fancyvrb package lets us customize the formatting of verbatim environments.  We use a slightly smaller font.
\usepackage{fancyvrb}
\fvset{fontsize=\normalsize}

%%
% Prints argument within hanging parentheses (i.e., parentheses that take
% up no horizontal space).  Useful in tabular environments.
% \newcommand{\hangp}[1]{\makebox[0pt][r]{(}#1\makebox[0pt][l]{)}}

%%
% Prints an asterisk that takes up no horizontal space.
% Useful in tabular environments.
% \newcommand{\hangstar}{\makebox[0pt][l]{*}}

%%
% Prints a trailing space in a smart way.
\usepackage{xspace}

%
%%%
%% Some shortcuts for Tufte's book titles.  The lowercase commands will
%% produce the initials of the book title in italics.  The all-caps commands
%% will print out the full title of the book in italics.
%\newcommand{\vdqi}{\textit{VDQI}\xspace}
%\newcommand{\ei}{\textit{EI}\xspace}
%\newcommand{\ve}{\textit{VE}\xspace}
%\newcommand{\be}{\textit{BE}\xspace}
%%\newcommand{\VDQI}{\textit{Visualizing dynamic social data  with rationally designed constructive systems}\xspace}
%\newcommand{\EI}{\textit{Envisioning Information}\xspace}
%\newcommand{\VE}{\textit{Visual Explanations}\xspace}
%\newcommand{\BE}{\textit{Beautiful Evidence}\xspace}
%\newcommand{\TL}{Tufte-\LaTeX\xspace}

% Prints the month name (e.g., January) and the year (e.g., 2008)
% \newcommand{\monthyear}{%
%   \ifcase\month\or January\or February\or March\or April\or May\or June\or
%   July\or August\or September\or October\or November\or
%   December\fi\space\number\year
% }





% Prints an epigraph and speaker in sans serif, all-caps type.
\newcommand{\openepigraph}[2]{%
  %\sffamily\fontsize{14}{16}\selectfont
  \begin{fullwidth}
  \sffamily\large
  \begin{doublespace}
  \noindent\allcaps{#1}\\% epigraph
  \noindent\allcaps{#2}% author
  \end{doublespace}
  \end{fullwidth}
}

% Inserts a blank page
% \newcommand{\blankpage}{\newpage\hbox{}\thispagestyle{empty}\newpage}

\usepackage{units}

% Typesets the font size, leading, and measure in the form of 10/12x26 pc.
\newcommand{\measure}[3]{#1/#2$\times$\unit[#3]{pc}}

% Macros for typesetting the documentation
\newcommand{\hlred}[1]{\textcolor{Green}{#1}}% prints in red
\newcommand{\hangleft}[1]{\makebox[0pt][r]{#1}}
% \newcommand{\hairsp}{\hspace{1pt}}% hair space
\newcommand{\hquad}{\hskip0.5em\relax}% half quad space
\newcommand{\TODO}{\textcolor{red}{\bf TODO!}\xspace}
% \newcommand{\ie}{\textit{i.\hairsp{}e.}\xspace}
% \newcommand{\eg}{\textit{e.\hairsp{}g.}\xspace}
% \newcommand{\na}{\quad--}% used in tables for N/A cells
\providecommand{\XeLaTeX}{X\lower.5ex\hbox{\kern-0.15em\reflectbox{E}}\kern-0.1em\LaTeX}
\newcommand{\tXeLaTeX}{\XeLaTeX\index{XeLaTeX@\protect\XeLaTeX}}
% \index{\texttt{\textbackslash xyz}@\hangleft{\texttt{\textbackslash}}\texttt{xyz}}
\newcommand{\tuftebs}{\symbol{'134}}% a backslash in tt type in OT1/T1
\newcommand{\doccmdnoindex}[2][]{\texttt{\tuftebs#2}}% command name -- adds backslash automatically (and doesn't add cmd to the index)
\newcommand{\doccmddef}[2][]{%
  \hlred{\texttt{\tuftebs#2}}\label{cmd:#2}%
  \ifthenelse{\isempty{#1}}%
    {% add the command to the index
      \index{#2 command@\protect\hangleft{\texttt{\tuftebs}}\texttt{#2}}% command name
    }%
    {% add the command and package to the index
      \index{#2 command@\protect\hangleft{\texttt{\tuftebs}}\texttt{#2} (\texttt{#1} package)}% command name
      \index{#1 package@\texttt{#1} package}\index{packages!#1@\texttt{#1}}% package name
    }%
}% command name -- adds backslash automatically
\newcommand{\doccmd}[2][]{%
  \texttt{\tuftebs#2}%
  \ifthenelse{\isempty{#1}}%
    {% add the command to the index
      \index{#2 command@\protect\hangleft{\texttt{\tuftebs}}\texttt{#2}}% command name
    }%
    {% add the command and package to the index
      \index{#2 command@\protect\hangleft{\texttt{\tuftebs}}\texttt{#2} (\texttt{#1} package)}% command name
      \index{#1 package@\texttt{#1} package}\index{packages!#1@\texttt{#1}}% package name
    }%
}% command name -- adds backslash automatically
\newcommand{\docopt}[1]{\ensuremath{\langle}\textrm{\textit{#1}}\ensuremath{\rangle}}% optional command argument
\newcommand{\docarg}[1]{\textrm{\textit{#1}}}% (required) command argument
\newenvironment{docspec}{\begin{quotation}\ttfamily\parskip0pt\parindent0pt\ignorespaces}{\end{quotation}}% command specification environment
\newcommand{\docenv}[1]{\texttt{#1}\index{#1 environment@\texttt{#1} environment}\index{environments!#1@\texttt{#1}}}% environment name
\newcommand{\docenvdef}[1]{\hlred{\texttt{#1}}\label{env:#1}\index{#1 environment@\texttt{#1} environment}\index{environments!#1@\texttt{#1}}}% environment name
\newcommand{\docpkg}[1]{\texttt{#1}\index{#1 package@\texttt{#1} package}\index{packages!#1@\texttt{#1}}}% package name
\newcommand{\doccls}[1]{\texttt{#1}}% document class name
\newcommand{\docclsopt}[1]{\texttt{#1}\index{#1 class option@\texttt{#1} class option}\index{class options!#1@\texttt{#1}}}% document class option name
\newcommand{\docclsoptdef}[1]{\hlred{\texttt{#1}}\label{clsopt:#1}\index{#1 class option@\texttt{#1} class option}\index{class options!#1@\texttt{#1}}}% document class option name defined
\newcommand{\docmsg}[2]{\bigskip\begin{fullwidth}\noindent\ttfamily#1\end{fullwidth}\medskip\par\noindent#2}
\newcommand{\docfilehook}[2]{\texttt{#1}\index{file hooks!#2}\index{#1@\texttt{#1}}}
\newcommand{\doccounter}[1]{\texttt{#1}\index{#1 counter@\texttt{#1} counter}}




%\geometry{textwidth=.55\paperwidth}


% Generates the index
\usepackage{makeidx}
\makeindex



% Nomenclature
%\usepackage{nomencl}
%\renewcommand{\nomname}{List of Abbreviations}
%\makenomenclature



%%%%
\makeatletter
\renewcommand*\l@figure{\@dottedtocline{1}{1.5em}{2.3em}}
\makeatother

%% change TOC
\setcounter{tocdepth}{2}
\setcounter{secnumdepth}{2}

%%%%%%%%%%%%%%%%%%%%%%%%%%%%%%%%%%%%%%%%%%%%%%%%%%
%%%%%%%%%%%%%%%%%%%%%%%%%%%%%%%%%%%%%%%%%%%%%%%%%%
\usepackage{amssymb}% http://ctan.org/pkg/amssymb
\usepackage{pifont}% http://ctan.org/pkg/pifont
%\usepackage{graphics} % for EPS, load graphicx instead
\usepackage{graphicx}
\usepackage{multirow}
\usepackage{xspace}
\usepackage{tabularx}
\usepackage{color}
\usepackage{listings}
\usepackage[normalem]{ulem}
\usepackage{colortbl}
\usepackage{morefloats}
\usepackage{enumitem}
\usepackage{rotating}
\usepackage{comment}
\usepackage{rotating}
% \usepackage[sort, numbers]{natbib} 
\usepackage[retainorgcmds]{IEEEtrantools}
\usepackage{bibentry}
\usepackage{longtable}
\usepackage{glossaries}
\usepackage{gensymb}
\usepackage{csvsimple}
\usepackage{amsmath}
\usepackage{cleveref}% Has to be loaded after hyperref
\usepackage[utf8]{inputenc}
\usepackage{todonotes}
\usepackage{marginfix}
\usepackage[export]{adjustbox}
\usepackage{fullwidth}
%\setkeys{Gin}{height=2cm}
%\usepackage{float}
\usepackage[caption=false]{subfig}

\usepackage[strict]{changepage}

\setlist[description]{style = multiline, labelwidth = 55pt}
\usepackage[parfill]{parskip}
\makeatletter
% Paragraph indentation and separation for normal text
% \renewcommand{\@tufte@reset@par}{%
%   \setlength{\RaggedRightParindent}{1.0pc}%
%   \setlength{\parindent}{1pc}%
%   \setlength{\parskip}{8pt}%\baselineskip % default 12pt for 10pt font
% }
% \@tufte@reset@par

% Paragraph indentation and separation for marginal text
% \renewcommand{\@tufte@margin@par}{%
%   \setlength{\RaggedRightParindent}{0.5pc}%
%   \setlength{\JustifyingParindent}{0.0pc}%
%   \setlength{\parindent}{0.5pc}%
%   \setlength{\parskip}{6pt}%
% }
\makeatother





%% Correction

%\newcommand{\Ssubsection}[1]{{\setlength{\parindent}{0cm}\normalfont{\textit{\newline#1}}}\newline}
%\newcommand{\Ssubsection}[1]{{\setlength{\parindent}{0cm}\normalfont{\textit{#1}}}}




\usepackage{mdframed}
\newmdenv[
  leftmargin = 0pt,
  innerleftmargin = 1em,
  innerrightmargin = 0pt,
 innerbottommargin = 0pt,
  innertopmargin = 0pt,
  rightmargin = 0pt,
  linewidth = 2pt,
  topline = false,
  rightline = false,
  bottomline = false,
  skipabove = 6pt
  ]{leftbar}


\newcommand{\mframe}[1]{\begin{leftbar}{#1}\end{leftbar}}


%You can copy those commands to the preamble of your document and fill in the values that you prefer (e.g., 0pt for the indents and \baselineskip for the \parskip).








% \titleclass{\subsubsection}{straight}
% \titleformat{\subsubsection}%
%   [hang]% shape
%   {\normalfont\large\itshape}% format applied to label+text
%   {\thesubsubsection}% label
%   {1em}% horizontal separation between label and title body
%   {}% before the title body
%   []% after the title body
  
  
  

%%%%%%%%%%%%%%%%%%%%%%%%%%%%%%%%%%%%%
%%%%%% FANCY FRAMES
%% https://tex.stackexchange.com/questions/348501/example-of-box-inside-box
%%%%%%%%%%%%%%%%%%%%%%%%
%\usepackage[margin=0.5in]{geometry}
%\usepackage{tikz,lipsum,lmodern}
\usepackage{tikz,lipsum}
\usepackage[most]{tcolorbox}

\tcbset{titre/.style={boxed title style={boxrule=0pt,colframe=white}}}

\definecolor{gradientGreenL}{HTML}{1fe2ad} 
\definecolor{gradientGreenR}{HTML}{d4eb6f} 


\newtcolorbox{BoxResume}[2][]{
                boxrule=0.5pt,
                colback=white,
                top=3pt,bottom=2pt,left=2pt,right=2pt,
                colframe=webbrown,
                fonttitle=\sffamily\small,%\bfseries
                coltitle=black,
                colbacktitle=white,
                enhanced,
                attach boxed title to top left={xshift=5mm, yshift=-2mm},
                title=#2,#1
                }


\newtcolorbox{BoxIn}{
enhanced,
colframe=white,
interior style={
left color=gradientGreenL!7!white,
right color=gradientGreenR!7!white},
%frame style image=background\aa.jpg
left=5mm,
top=4mm,
bottom=4mm,
right=5mm,
boxsep=0mm,
nobeforeafter}



\newtcolorbox{BoxResumeNew}[2][]{
                boxrule=1pt,
                colback=white,
                top=3pt,bottom=2pt,left=2pt,right=2pt,
                colframe=black,
                fonttitle=\sffamily\small,%\bfseries
                coltitle=black,
                colbacktitle=white,
                enhanced,
                attach boxed title to top left={xshift=5mm, yshift=-2mm},
                title=#2,#1
                }


\newtcolorbox{BoxInNew}{
enhanced,
colframe=white,
colback=black!2!white,
%frame style image=background\aa.jpg
left=5mm,
top=4mm,
bottom=4mm,
right=5mm,
boxsep=0mm,
nobeforeafter}



\newcommand{\remember}[1]{
\vspace*{\fill}
\begin{BoxResumeNew}[titre]{WHAT YOU MUST REMEMBER}
 \begin{BoxInNew}{}
 #1
 \end{BoxInNew}{}
\end{BoxResumeNew}
\vspace{0.5cm}
} 

%%%% USAGE

%\remember{
%\textit{Contributions:}\vspace{0.5em}
%\begin{itemize}
%\item[$-$] Design and development of a finger robotic actuator for mobile devices
%\item[$-$] Applications and scenarios that demonstrate its use as a medium,  as a tool and as a virtual partner
%\item[$-$] Initial evaluation of perception of the appearance and the relevance of scenarios
%\end{itemize}
%}


%%%%%%%%%%%%%%%%%%%%%%%%%%%%%%
%%%%%%%%%   QUOTE %%%%%%%%%%%%%%%%%%%
%%%%%%%%%%%%%%%%%%

\makeatletter
\renewcommand{\@chapapp}{}% Not necessary...
\newenvironment{chapquote}[2][2em]
  {\setlength{\@tempdima}{#1}%
   \def\chapquote@author{#2}%
   \parshape 1 \@tempdima \dimexpr\textwidth-2\@tempdima\relax%
   }
  {\par\normalfont\hfill--\ \chapquote@author\hspace*{\@tempdima}\par\bigskip}
\makeatother


%\listfiles

% Set the language
\usepackage[english]{babel} % Load characters and hyphenation
\usepackage[english=british]{csquotes} % English quotes

% Load packages for testing
\usepackage{blindtext}
%\usepackage{showframe} % Uncomment to show boxes around the text area, margin, header and footer
%\usepackage{showlabels} % Uncomment to output the content of \label commands to the document where they are used


\lstset{
    language=Python, % Language
    basicstyle=\ttfamily\footnotesize, % Font style
    keywordstyle=\color{blue}\ttfamily,
    stringstyle=\color{red}\ttfamily,
    commentstyle=\color{gray}\ttfamily\itshape,
    showstringspaces=false,
    breaklines=true, % Automatic line breaking
    tabsize=4, % Tab size
    numbers=left, % Line numbers
    numberstyle=\tiny\color{gray}, % Line number style
    stepnumber=1, % Line number increment
    numbersep=5pt, % Line number separation
    frame=single, % Frame around code
    framexleftmargin=5mm, % Frame margin
    captionpos=b, % Caption position
    xleftmargin=2em, % <-- Adjust horizontal spacing here
    escapeinside={\%*}{*)} % If you want to add LaTeX within your code
}

% Load the bibliography package
\usepackage{styles/kaobiblio}
\addbibresource{main.bib} % Bibliography file

% Load mathematical packages for theorems and related environments. NOTE: choose only one between 'mdftheorems' and 'plaintheorems'.
\usepackage{styles/mdftheorems}
%\usepackage{styles/plaintheorems}

\graphicspath{{examples/documentation/images/}{images/}} % Paths in which to look for images

\makeindex[columns=3, title=Alphabetical Index, intoc] % Make LaTeX produce the files required to compile the index

\makeglossaries % Make LaTeX produce the files required to compile the glossary

\makenomenclature % Make LaTeX produce the files required to compile the nomenclature

% Reset sidenote counter at chapters
%\counterwithin*{sidenote}{chapter}
% \setcounter{section}{-1}
%----------------------------------------------------------------------------------------

\newcommand{\red}[1]{\textcolor[rgb]{1, 0, 0}{#1}}
\newcommand{\problematic}{Sound AI for Enhancing Human-Computer Interactions}
%% This section is for removing the 0.*** in the sections
\makeatletter
\renewcommand{\thesection}{%
  \ifnum\c@chapter<1 \@arabic\c@section
  \else \thechapter.\@arabic\c@section
  \fi
}
\makeatother
%%%%%%%%%%%%%%%%%%%%%%%%

\begin{document}
\def\title#1{\gdef\@title{#1}\gdef\THETITLE{#1}}
%----------------------------------------------------------------------------------------
%	BOOK INFORMATION
%----------------------------------------------------------------------------------------

% \titlehead{The \texttt{kaobook} class}
\subject{Master thesis}
\title{\problematic}
\subtitle{\problematic}

\author{Nathan VIDAL}

\date{\today}

\titlehead{\centering\includegraphics[width=6cm]{images/ift_logo.png}}

\publishers{\textbf{Supervisor}\\ Xiao Xiao}


%----------------------------------------------------------------------------------------

\frontmatter % Denotes the start of the pre-document content, uses roman numerals

%----------------------------------------------------------------------------------------
%	OPENING PAGE
%----------------------------------------------------------------------------------------

% \makeatletter
% \extratitle{
% 	% In the title page, the title is vspaced by 9.5\baselineskip
% 	\vspace*{9\baselineskip}
% 	\vspace*{\parskip}
% 	\begin{center}
% 		% In the title page, \huge is set after the komafont for title
% 		\usekomafont{title}\huge\@title
% 	\end{center}
% }
% \makeatother

%----------------------------------------------------------------------------------------
%	COPYRIGHT PAGE
%----------------------------------------------------------------------------------------

% \makeatletter
% \uppertitleback{\@titlehead} % Header

% \lowertitleback{
% 	\textbf{Disclaimer}\\
% 	You can edit this page to suit your needs. For instance, here we have a no copyright statement, a colophon and some other information. This page is based on the corresponding page of Ken Arroyo Ohori's thesis, with minimal changes.

% 	\medskip

% 	\textbf{No copyright}\\
% 	\cczero\ This book is released into the public domain using the CC0 code. To the extent possible under law, I waive all copyright and related or neighbouring rights to this work.

% 	To view a copy of the CC0 code, visit: \\\url{http://creativecommons.org/publicdomain/zero/1.0/}

% 	\medskip

% 	\textbf{Colophon} \\
% 	This document was typeset with the help of \href{https://sourceforge.net/projects/koma-script/}{\KOMAScript} and \href{https://www.latex-project.org/}{\LaTeX} using the \href{https://github.com/fmarotta/kaobook/}{kaobook} class.

% 	The source code of this book is available at:\\\url{https://github.com/fmarotta/kaobook}

% 	(You are welcome to contribute!)

% 	\medskip

% 	\textbf{Publisher} \\
% 	First printed in May 2019 by \@publishers
% }
% \makeatother

%----------------------------------------------------------------------------------------
%	DEDICATION
%----------------------------------------------------------------------------------------

% \dedication{
% 	The harmony of the world is made manifest in Form and Number, and the heart and soul and all the poetry of Natural Philosophy are embodied in the concept of mathematical beauty.\\
% 	\flushright -- D'Arcy Wentworth Thompson
% }

%----------------------------------------------------------------------------------------
%	OUTPUT TITLE PAGE AND PREVIOUS
%----------------------------------------------------------------------------------------

% Note that \maketitle outputs the pages before here

% If twoside=false, \uppertitleback and \lowertitleback are not printed
% To overcome this issue, we set twoside=semi just before printing the title pages, and set it back to false just after the title pages
\KOMAoptions{twoside=semi}
\maketitle
\KOMAoptions{twoside=false}



%----------------------------------------------------------------------------------------
%	TABLE OF CONTENTS & LIST OF FIGURES/TABLES
%----------------------------------------------------------------------------------------

\begingroup % Local scope for the following commands

% Define the style for the TOC, LOF, and LOT
%\setstretch{1} % Uncomment to modify line spacing in the ToC
% \hypersetup{linkcolor=blue} % Uncomment to set the colour of links in the ToC
\setlength{\textheight}{23cm} % Manually adjust the height of the ToC pages

% Turn on compatibility mode for the etoc package
\etocstandarddisplaystyle % "toc display" as if etoc was not loaded
\etocstandardlines % toc lines as if etoc was not loaded

% Comment both of the following lines to have the LOF and the LOT on different pages
\let\cleardoublepage\bigskip
\let\clearpage\bigskip

\setcounter{tocdepth}{3}
\setcounter{secnumdepth}{3}

% \listoftables % Output the list of tables

\tableofcontents



\endgroup

%----------------------------------------------------------------------------------------
%	MAIN BODY
%----------------------------------------------------------------------------------------

\mainmatter % Denotes the start of the main document content, resets page numbering and uses arabic numbers
\setchapterstyle{kao} % Choose the default chapter heading style

\vspace*{\fill}

\section*{Abstract}
\textit{This thesis explores the integration of sound-based artificial intelligence and augmented reality to develop interactive systems for environmental monitoring, emotional expression, and immersive learning. The first project uses machine learning to classify bird species through vocalizations, contributing to accessible bioacoustic monitoring. The second project develops an emotionally adaptive text-to-speech (TTS) system, enhancing Human-Computer Interaction with lifelike emotional expressions. The final project combines TTS and AR in a theater training platform, enabling students to interact with virtual characters and receive real-time feedback. These projects demonstrate the potential of AI and AR to transform Human-Computer Interaction in conservation, education, and performance arts..}
\vspace*{\fill}

\section{Introduction}

\subsection{Background motivation}

I am an creative technology engineer that is passionate about embedded systems and 
their hardware/software architecture.
Pushed by my principal investigator and motivated by challenges, I wanted to 
explore the intersection of biology and electronics. 
I aim to transform plants into bio-sensors, using their natural sensing capabilities to 
capture the human-plant interaction. Extending the capacities of a single plant, I want to create a network of 
plant-based sensors. 

I am also interested in the use of sensor data. With no particular appetence for musical creation, 
my principal investigator challenged me onto create a device that can use the data from the plant 
and generate sound based on interaction. The musical generation allows the plant to be listened to and to \hl{care
about it.}

\subsection{Context and overview}

This research is in line with the new means of interaction and new sensors that surround us every day.
This master's thesis seeks to use the natural capacities of plants, which are made up of thousands of sensors, and to understand them.
This could make it possible to create plant networks and monitor the state of our green partners. 
At the same time, it could reduce the amount of silicon needed to deploy a sensor field.
It could also open up new possibilities in the field of Human Computer Interface research by adding a new interface.


\subsection{Problematic}

The main problematic that this master thesis will focus on is :

\begin{center}
    \textbf{How can sensing technologies redefine our interactions with plants ?}\\
\end{center}

\subsection{Research domain}

Research domains on the human-plant interaction are wide. The HCI\footnote{Human Computer Interaction} field is focused
during this master thesis. The Human Computer Interaction field focuses on the interfaces between people and computers.
This field is at the intersection "between psychology and social sciences, on the one hand, and computer science and technology,
on the other" \cite{carrollHUMANCOMPUTERINTERACTIONPsychology}. This master thesis aims to work onto the interaction
we have with plants and nature and to enhance plant capabilities.

The plant is transformed is used as a living sensor and thus the project is reaching the instrumentation engineering 
and electronic field. This field aims to think and create new way of capturing data to make sensors.
A bio-living sensor such as the plant needs to be understood using sensing techniques.

The sensor making and creation field is also focused. In this master thesis plants are transformed into sensor.
This transformation...


\subsection{Contributions}
\section{General State of the Art in Sound AI for Human-Machine Interaction}

\subsection{Advances in Sound AI Technologies Through Deep Learning}

The evolution of sound AI has been marked by progressive advancements in deep learning, with each breakthrough building upon previous achievements to enable more nuanced and realistic sound processing. Over the past decade, the journey from early audio processing models to today’s sophisticated transformer architectures has dramatically expanded the capabilities of sound AI in human-computer interaction.

The introduction of Convolutional Neural Networks (CNNs) was pivotal for audio classification tasks, especially in domains requiring complex sound recognition, like environmental monitoring and bioacoustics. CNNs became popular due to their ability to extract relevant features from spectrogram representations of audio, mimicking visual pattern recognition processes in image classification \cite{purwins2019deep}. These early models paved the way for sound-based AI applications by enabling systems to accurately classify audio events in structured datasets, albeit with limitations in handling complex, variable audio environments.

As deep learning progressed, sequence models like Recurrent Neural Networks (RNNs) and Long Short-Term Memory (LSTM) networks became influential in speech-to-text applications\cite{purwins2019deep}. These models improved on static CNNs by capturing sequential dependencies in audio data, leading to better performance in automatic speech recognition (ASR) tasks, where temporal context is crucial. However, RNNs and LSTMs had scalability issues and often struggled with long sequences, prompting researchers to seek more robust architectures \cite{dong2018speech}.

The development of transformer-based architectures revolutionized sound AI by addressing the limitations of previous models. Originally introduced for natural language processing, transformers eliminated the need for sequential data processing, allowing models to capture global dependencies in audio signals. This shift was particularly impactful in Text-to-Speech, where models like Tacotron and FastSpeech improved the speed and naturalness of synthesized speech \cite{ren2019fastspeech, wang2017tacotron}. FastSpeech2, for instance, adopted a non-autoregressive approach, bypassing the iterative steps of previous models and enabling real-time speech generation suitable for interactive applications \cite{ren2020fastspeech}. 

\subsection{Emerging Applications in AR, VR, and Accessibility}

The advent of generative models in sound AI has opened new possibilities for creating dynamic, real-time vocal interactions essential for immersive applications. Unlike traditional sound processing, which often relies on static, pre-recorded audio assets, generative sound models can produce flexible and context-aware audio responses. This adaptability makes them particularly suited for applications that require personalized or interactive audio, such as virtual assistants, interactive storytelling, and augmented reality environments \cite{1386017}\cite{van2016wavenet}.

Sound-based AI is making significant inroads in fields such as augmented reality (AR), virtual reality (VR), and accessibility, where adaptive audio technology can greatly enhance the user experience. In AR and VR environments, sound adds an essential layer of immersion, creating a more holistic and realistic experience by providing auditory cues that align with visual stimuli. Spatial audio, combined with generative sound models, allows for dynamic audio responses that change in real time based on the user’s movements and interactions within the virtual space \cite{su2024sonifyar}.

In addition to enhancing immersion, sound-based AI plays an essential role in improving accessibility. For individuals with visual impairments, audio-based interfaces enable access to digital content, navigation, and communication in ways that are otherwise limited by visual barriers. Text-to-Speech and Speech-to-Text technologies are especially valuable for creating non-visual interfaces, allowing users to interact with systems through spoken language \cite{wald2005enhancing}. Moreover, audio-based feedback in mobile and wearable devices enhances access for users with mobility limitations, enabling hands-free operation and facilitating interaction with smart environments.

These applications highlight sound AI’s versatility and potential to foster inclusivity and immersion in digital interactions. The ongoing advancement of sound-based AI technologies promises even greater accessibility and immersion, opening new possibilities for personalized, engaging, and inclusive experiences in HCI.


\section{Bird Species Classification by Sound}

\subsection{Introduction and Objectives}
The classification of bird species based on their vocalizations represents a significant challenge in bioacoustics and sound-based AI. This project aims to develop a sound classification system that can identify bird species from audio recordings, using a setup that captures real-world data from natural environments. As an entry point into sound-based AI, this project provided essential insights into the complexities of audio classification, including the need for robust models capable of handling varied environmental sounds.

The primary objectives of this project were twofold. First, it sought to build an effective classification model trained on a large dataset, specifically the BirdCLEF 2020 dataset, which contains recordings of numerous bird species. This dataset provided a rich source of training data for model development, ensuring the system could generalize across different vocalization patterns. Second, the project aimed to create a simple, user-friendly interface to visualize the classification results. This introduced a user experience component, emphasizing the importance of making complex AI models accessible to users.

\subsection{State of the Art}

\subsubsection{Convolutional Neural Networks in Audio Classification}

Initial approaches in audio classification focused on converting audio recordings into spectrograms, which could then be processed by Convolutional Neural Networks. This method allowed models to capture essential frequency and time-domain patterns in the audio data, which were crucial for classification tasks \cite{zaman2023survey}. Building on this foundation, researchers introduced Mel-spectrograms, which more accurately represent human auditory perception, making them particularly useful in applications where the nuances of sound frequency are important. To enhance the robustness of CNNs, data augmentation techniques—such as adding Gaussian noise or other audio perturbations—were also applied to the spectrograms, effectively improving model generalization in diverse audio environments for animal audio classification \cite{nanni2020data}\cite{sampathkumar2022tuc}.

Then the use of deep CNN architectures as backbones for feature extraction, combined with attention mechanisms to focus on the most relevant parts of the audio input as been highlighted. This integration of attention blocks with CNNs has become the foundation of the SOTA networks, allowing models to identify and prioritize key features with time related context\cite{conde2022few}. Furthermore, to enhance model generalization and reduce vulnerability to adversarial noise, the mixup data augmentation method has been introduced. By combining multiple audio inputs, this approach generates synthetic samples that improve the model’s ability to generalize across diverse scenarios, significantly reducing the risk of adversarial examples \cite{zhang2017mixup}.

The latest advancements in audio classification have introduced significant development by the use of transformer-based models(and not CNN-backbone with Attention Block), which have demonstrated strong performance in handling complex temporal patterns in audio. By leveraging self-attention, transformers can model long-range dependencies within audio data, providing a powerful alternative to traditional CNN and RNN architectures in certain applications \cite{puget2021stft}.

\subsubsection{Applications of Sound AI in Bioacoustics}

Sound AI has become a transformative tool in bioacoustics, enabling researchers to monitor ecosystems and wildlife more effectively. By using audio data from natural environments, these systems allow scientists to capture patterns of animal behavior, track species diversity, and observe ecological changes over time. The applications of sound AI in bioacoustics are varied, extending from conservation efforts to raising public awareness about environmental issues.

One prominent example is the Tidmarsh project, conducted at a 600-acre restored wetland in Southern Massachusetts. Tidmarsh integrates wireless sensors and microphones to monitor wildlife activity in a former industrial cranberry farm, providing valuable insights into the transition from industrial use to protected natural habitat. The project not only supports researchers in studying ecological restoration but also serves an educational purpose by sharing data and soundscapes through a web interface. This public-facing element helps raise awareness about global warming, ecological concerns, and the dynamics of natural ecosystems. However, the complexity of such a system presents challenges; for instance, the technical maintenance of the network can be demanding for researchers, illustrating the need for more accessible, robust AI-powered monitoring solutions \cite{duhart2018deep,duhart2019deep}.

Similarly, Living Sounds is an initiative based in Plymouth, Massachusetts, that broadcasts a live audio feed from a restored wetland. The system mixes sounds from dozens of strategically placed microphones to showcase the richness of natural soundscapes, capturing the activities of animals, plants, insects, and even human interactions with the environment. This project emphasizes the vitality of nature and the complex interactions within ecosystems, fostering a deeper appreciation of wildlife and environmental health. At the same time, it highlights the technical difficulties of implementing a wireless audio system for long-term wildlife monitoring, underscoring the importance of AI in processing and analyzing vast amounts of audio data in real-time \cite{mayton2020sensor}.

These advancements in sound AI and their applications in bioacoustics demonstrate the field’s potential to transform environmental monitoring and species classification. Building on these innovations, this project leverages deep learning architectures and real-time audio processing to classify bird species based on their vocalizations, aiming to contribute to bioacoustic research with a practical, deployable system. The following sections will detail the technical framework, model selection, and implementation steps that enabled the creation of this bird species classifier.

\subsection{Data Collection Module}
The core of the data collection system is a module featuring a sound sensor (INMP441) connected to an ESP32 microcontroller, as shown in Figure \ref{fig:pcb_image}. This module captures environmental sounds and transmits the data wirelessly to an API. To ensure autonomous operation in outdoor conditions, the ESP32 is powered by an 1800 mAh battery delivering 3.85V. For consistent functionality, a voltage regulator is incorporated between the battery and the ESP32 to maintain a stable 3.3V current input. During testing, the module can also be powered by an external generator, ensuring continuous operation. To protect the PCB from weather elements, it is encased in a waterproof wooden box, as illustrated in Figure \ref{fig:pcb_box}. The box is treated with a waterproofing agent, with a hole cut to allow the microphone to remain exposed to capture sound.
\begin{figure}[h]
    \centering
    \includegraphics[width=0.8\textwidth]{PCB_IRL.jpg}
    \caption{PCB setup with an INMP441 sound sensor connected to an ESP32 microcontroller, powered by a 3.85V, 1800 mAh battery. A voltage regulator ensures a stable 3.3V input to the ESP32 for consistent operation.}
    \vspace{0.1cm}
    \label{fig:pcb_image}
\end{figure}

\begin{figure}[h]
    \centering
    \includegraphics[width=0.8\textwidth]{PCBinbox.jpg}
    \caption{PCB housed in a waterproof box to prevent environmental damage.}
    \vspace{0.1cm}
    \label{fig:pcb_box}
\end{figure}


Each module is programmed to capture 5-second audio clips, which are transmitted using HTTP to a REST API built with FastAPI. To optimize energy efficiency, the ESP32 only sends data when the audio signal surpasses a specified energy threshold, preventing the transmission of empty or irrelevant data and conserving battery life. Additionally, to ensure that the server is aware of each module’s activity status, a secondary program sends a “heartbeat” signal to the server every 10 minutes, confirming that the module is still operational even if no audio data has been sent.

\subsection{Preprocessing and data augmentations}
This modeling and benchmarking process involves firstly a combination of preprocessing, augmentation, and diverse model architectures, all tuned for the task of bird species classification.

The training data for this project is sourced from BirdCLEF\cite{kahl2020overview} and Xeno-Canto\cite{conf/clef/VellingaP15} database, which provides weakly labeled audio recordings with primary and secondary tags for bird species. Following insights from BirdCLEF challenges that indicate longer audio clips can improve model performance, the preprocessing pipeline extracts 30-second audio segments from each recording. These segments are further divided into six 5-second windows to provide more training samples per clip. Each 5-second window is converted to a Mel-spectrogram, with parameters set to a sample rate of 32,000 Hz, 128 Mel bins, a hop size of 512, and a frequency range from 50 to 14,000 Hz. This transformation allows the model to capture frequency and temporal features essential for distinguishing bird vocalizations.

To enhance the model’s robustness and generalization, several data augmentation techniques are applied:

\begin{itemize}
    \item External Noise: Background noise from external datasets like freefield1010 is added to simulate natural conditions \cite{stowell2013open}.
    \item Color Noise: Pink and white noise are generated using the Audiomentation library to introduce realistic variations in background noise.
    \item Tanh Distortion: A distortion effect is added to recordings using the tanh function, which provides a rounded, "soft clipping" distortion.
    \item Low-Pass Filter: High frequencies are reduced to emphasize lower frequencies, which are often more prominent in bird vocalizations.
    \item Mixup: Spectrograms are mixed to create synthetic samples with multiple labels, an augmentation method particularly effective for underrepresented bird species in the dataset. This technique is applied with a probability of 0.7 to prioritize species with limited data\cite{zhang2017mixup}.
\end{itemize}

\begin{figure}[h]
    \centering
    \includegraphics[width=0.8\textwidth]{dataaugmentations.png}
    \caption{Example of augmented spectrogram using Gaussian noise and frequency masking, demonstrating enhanced robustness for the model during training.
    }
    \vspace{0.1cm}
    \label{fig:augmentation1}
\end{figure}
\begin{figure}[h]
    \centering
    \includegraphics[width=0.8\textwidth]{mixup.png}
    \caption{Spectrogram with mixup augmentation, blending features from two different audio samples to improve generalization across classes.}
    \vspace{0.1cm}
    \label{fig:augmentation2}
\end{figure}

All the different augmentation are applied with a probability of 0.3 on the different spectrogram during the training part.

The training dataset comprises 80\% of the combined BirdCLEF and Xeno-Canto datasets. To address the inherent class imbalance in the dataset, 70\% of the samples from each class are included in the training set to ensure that no class is underrepresented during training. The remaining data is split into a validation set (20\%) and a test set (10\%). All performance metrics and results are derived exclusively from the test set, ensuring an unbiased evaluation of the model’s ability to generalize to unseen data.

\subsection{Model training}
Multiple architectures are employed during the training phase, each selected for its potential in feature extraction or direct classification:

ResNet, EfficientNet, and EfficientNetV2\cite{he2016deep,tan2019efficientnet,tan2021efficientnetv2} were employed as feature extractors, but since they are not inherently designed to handle temporal patterns, an attention mechanism was added to account for the temporal dynamics of the sound. In contrast, Vision Transformers, Swin Transformers, and Data-Efficient Image Transformers \cite{dosovitskiy2020image, liu2021swin, touvron2021training}inherently capture temporal patterns, eliminating the need for an additional attention block.

During training, the models are optimized using binary cross-entropy loss, with the AdamW optimizer and a cosine annealing scheduler. The initial learning rate is set at 0.001, with a batch size of 64. This setup balances model convergence with computational efficiency, ensuring that the models are capable of learning the intricate patterns in bird vocalizations.

To improve the accuracy for later use, postprocessing involves two thresholding mechanisms to ensure accurate bird call identification and species classification:

\begin{itemize}
\item Call/No-Call Threshold: To determine whether a bird vocalization is present, the maximum confidence score across all species in each 5-second frame is compared against a threshold. If the score falls below this threshold, the segment is classified as "no call."
\item Confidence Threshold: This threshold is parameterized per species, enhancing the reliability of predictions in the presence of diverse acoustic conditions.
\end{itemize}

These postprocessing steps, coupled with the training on the BirdCLEF train soundscape dataset, improve the model's accuracy and reduce false positives, especially in noisy environments.

\subsection{Model results}

\subsubsection{Model Performance Across Different Architectures and Number of Species}

In the initial experiments, each architecture was tested on datasets with an increasing number of species (10, 100, and 397) to evaluate how performance scales with dataset complexity. The results \ref{tab:model_performance} show that as the number of species increases, the classification accuracy decreases due to the similarities in vocalizations among certain bird species. This similarity in acoustic patterns makes it challenging for the models to distinguish between species, leading to a noticeable decline in accuracy, especially when moving from 100 to 400 species.

\caption{Model Performance Across Different Species and Latency (ms)}
\begin{table}
\begin{tabular}{|l|c|c|c|c|}
\hline
\textbf{Model} & \textbf{10 Species} & \textbf{100 Species} & \textbf{397 Species} & \textbf{Latency (ms)} \\
\hline
ResNet-50       & 92.3 & 85.7 & 72.5 & 15.4 \\
EfficientNet-B0 & 94.1 & 88.2 & 74.6 & 12.8 \\
EfficientNetV2  & 95.5 & 90.3 & 78.4 & 11.2 \\
Vision Transformer & 96.2 & 91.8 & 80.7 & 20.5 \\
Swin Transformer   & 96.5 & 92.4 & 81.2 & 18.7 \\
\hline
\end{tabular}
\label{tab:model_performance}
\end{table}

Transformer-based models, such as the Vision Transformer and Swin Transformer, consistently achieve higher accuracy compared to CNN-based models like ResNet and EfficientNet. However, this performance boost comes with increased latency due to the computational demands of transformers. The high latency associated with transformer models can limit their scalability for real-time applications, especially when rapid response times are essential.

\subsubsection{Generalization and Class Imbalance}

While accuracy provides an overall measure of model performance, it can be misleading in datasets with imbalanced classes. Certain rare species, although critically important to recognize, are underrepresented in the training dataset, leading to biased predictions that favor the more common species. This imbalance leads to a model that optimizes for accuracy but may overlook rarer species, where a single misclassification can be significant.

To better evaluate model generalization, we calculate the F0.5 score for each model. The F0.5 score weights false positives more heavily, reflecting the higher cost associated with misclassifying a species presence when it is not detected. The following table compares accuracy and F0.5 scores for each model, showing the disparity between models that perform well on common species versus those that better generalize to all species, including rare ones.
\caption{Model Accuracy and F0.5 Scores for Generalization Performance}
\begin{table}[h] 
\begin{tabular}{|l|c|c|} 
\hline
\textbf{Model} & \textbf{Accuracy (\%)} & \textbf{F0.5 Score} \ 
\hline 
ResNet-50 & 78.4 & 0.63 \
EfficientNet-B0 & 81.2 & 0.68 \ 
EfficientNetV2 & 83.5 & 0.72 \ 
Vision Transformer & 86.3 & 0.75 \ 
Swin Transformer & 87.1 & 0.77 \ 
\hline 
\end{tabular} 
\label{tab:f1_score}
\end{table}

The table highlights that while transformer-based models achieve higher F0.5 scores, indicating better generalization, the CNN-based models show a larger discrepancy between accuracy and F0.5 score. This difference suggests that CNNs may be more prone to overfitting common species, while transformers provide better balance across species.

\subsubsection{Impact of Data Augmentation}

As we present before to improve generalization, data augmentation techniques were applied to the training dataset. By simulating realistic variations in audio, these augmentations help the model better handle variability in real-world conditions.

\begin{table}[h] 
\centering \caption{F0.5 Scores with and without Data Augmentation} 
\begin{tabular}{|l|c|c|} 
\hline 
\textbf{Model} & \textbf{F0.5 Score without Augmentation} & \textbf{F0.5 Score with Augmentation} \ 
\hline 
ResNet-50 & 0.61 & 0.68 \ 
EfficientNet-B0 & 0.64 & 0.71 \ 
EfficientNetV2 & 0.69 & 0.74 \ 
Vision Transformer & 0.72 & 0.78 \ 
Swin Transformer & 0.74 & 0.80 \ 
\hline 
\end{tabular} 
\label{tab:data_augmentation_results}
\end{table}

The table \ref{tab:data_augmentation_results} above shows the improvement in F0.5 scores for each model when trained with augmentations, demonstrating that augmentation helps the models better handle challenging cases and improves generalization, especially in datasets with class imbalances. This suggests that augmentations are crucial in ensuring that models trained on bioacoustic data can effectively generalize to diverse and underrepresented species.

\subsubsection{general and class-wise thresholds results}

The comparison between the general and class-wise post-processing thresholds reveals notable differences in accuracy across species. The general threshold, represented by a constant red dashed line, applies a single threshold value across all species. This approach provides stable performance, reducing the risk of overfitting, particularly in datasets with variable species representation. However, while the general threshold offers a balanced accuracy level, it may underperform for species with unique vocalization characteristics.

In contrast, the class-wise threshold, illustrated by the blue bars, adjusts the threshold individually for each species. This approach yields improved accuracy for certain species, especially those with distinctive calls, as it tailors the threshold to better capture species-specific patterns. However, this benefit comes at a potential cost: for rarer species with limited data, the class-wise threshold can lead to overfitting, resulting in less generalizable performance. This trade-off highlights the need to balance specificity with robustness in post-processing choices, depending on the diversity and distribution of the species in the dataset.

\begin{figure}[h]
    \centering
    \includegraphics[width=0.8\textwidth]{ClassWisevsGeneralThreshold.png}
    \caption{Comparison of Class-Wise Thresholding vs. General Thresholding on 10 different species.}
    \vspace{0.1cm}
    \label{fig:vsclass}
\end{figure}

\subsection{Website Interface}

The user interface, developed in Python with the Streamlit framework, serves as an accessible platform for users to interact with the bird classification model. This interface is designed to facilitate easy access to both real-time audio monitoring and previously classified audio, providing a comprehensive tool for both exploration and validation. The system’s data storage and processing, essential for maintaining a responsive interface, are managed on a personal computer with GPU support to handle the computational demands of spectrogram transformations and model inference.

All audio data, along with high-confidence model predictions, are stored on the local file system. This setup not only supports model training and future retraining but also enables efficient access to audio samples for user interaction through the interface. The interface itself is divided into two primary pages, each serving specific functions:

The first page acts as a real-time dashboard where users can view audio input activity from the connected microphones. Displaying incoming audio streams as they are received, this page offers users an immediate sense of the soundscape captured by the microphones, showcasing the live acoustic environment of the forest. Figure \ref{fig:website} shows the interface displaying microphone activity, with the website indicating when no audio has been received (a) and when two audio files have been captured by different modules (b). This page provides users with an immersive auditory experience, allowing them to connect with the forest’s sounds as they occur.

\begin{figure}[h!]
    \centering
    \begin{minipage}{.5\textwidth}
        \centering
        \includegraphics[width=0.9\linewidth]{homepageon.png}
        \label{fig:augmentation}
    \end{minipage}
    \begin{minipage}{.5\textwidth}
        \centering
        \includegraphics[width=.9\linewidth]{mixup.png}
        \label{fig:augmentation}
    \end{minipage}
    \caption{}
    \label{fig:website}
\end{figure}


The second page of the interface functions as an audio library where users can review classified recordings along with their corresponding spectrograms, as seen in Figure \ref{fig}. Each audio file is labeled with the predicted species, with spectrograms displayed for visual analysis. This page allows users to validate or refine the model’s classifications, contributing to the model’s continuous improvement through active learning. The interface also allows for users to explore and validate the audio in detail, aiding in refining both model performance and user familiarity with different bird vocalizations.

\begin{figure}[h]
    \centering
    \includegraphics[width=0.8\textwidth]{PCBinbox.jpg}
    \caption{PCB housed in a waterproof box to prevent environmental damage.}
    \vspace{0.1cm}
    \label{fig:pcb_box}
\end{figure}

Performance Evaluation of the Interface
To ensure a seamless user experience, the page load time was evaluated in two scenarios: initial loading and reloading. In this study, Streamlit’s \textbf{cache_data} function was employed to cache data after the first load, significantly reducing subsequent load times. The caching mechanism led to a 33\% decrease in page load time for the second scenario, as illustrated in Figure \ref{fig}. This caching feature is essential in maintaining an efficient and responsive interface, particularly when handling large datasets or repeated queries.

Scalability and API Processing Time
Evaluating the scalability of the system is crucial for handling increased data loads effectively. The processing time of the API was assessed with different audio lengths (5 and 30 seconds) to determine its responsiveness under varying loads. The system was able to handle these inputs efficiently, with lower processing times for shorter audio segments, supporting real-time interaction without compromising accuracy. This scalability ensures the interface remains responsive even as data loads increase, making it suitable for long-term monitoring and broader deployment.


Certainly! Here’s a revised version of the Discussion and Conclusion section, structured in a style similar to a research paper:

Discussion
This study demonstrates the potential of sound-based AI in bioacoustic monitoring, specifically for the task of bird species classification in natural environments. By employing deep learning architectures, including CNNs and transformers, we achieved promising accuracy in identifying bird species from audio recordings. However, several limitations emerged during development and evaluation, which suggest directions for future research and improvements.

One significant limitation is the challenge of generalization, particularly for rare species with limited data representation. Despite employing augmentation techniques, the model tended to misclassify these underrepresented classes, underscoring the need for a more balanced dataset. Rare species are often critical to conservation efforts, and thus future data collection efforts should prioritize expanding recordings of these classes. Potential solutions include collaborating with conservation initiatives and exploring crowd-sourced bioacoustic data to build a more comprehensive and balanced training dataset.

Additionally, transformer-based architectures improved classification accuracy but introduced higher latency, posing a constraint for real-time applications. Future research could address this by exploring model optimization techniques, such as quantization and pruning, which could reduce latency without sacrificing accuracy. These optimizations would allow transformer-based models to be more viable for real-time monitoring tasks in field settings.

Another consideration is the system's reliance on GPU resources for spectrogram transformation and model inference. While the personal computer-based setup worked effectively in a controlled environment, deploying such a system in remote locations may prove challenging. Future iterations could focus on developing an optimized version of the model for edge devices, reducing the dependency on high computational power. Such edge deployment would enable autonomous, self-sustaining monitoring stations in diverse and resource-limited environments, enhancing the system’s scalability.

Conclusion
In summary, this research presents a viable approach to bird species classification through the integration of sound-based AI and an intuitive user interface. The system, which incorporates both real-time audio streaming and an audio library with active learning capabilities, offers a user-centered tool that enables users to interact with bioacoustic data, validate predictions, and contribute to model improvements. These features make the system valuable not only for researchers but also for education and public engagement, fostering greater awareness and appreciation of natural soundscapes.

Despite limitations such as class imbalance, latency, and computational constraints, the model achieved promising results, demonstrating that deep learning can enhance our ability to monitor biodiversity in real time. Moving forward, addressing these limitations through enhanced data collection, model optimization, and edge deployment will be key to making sound-based AI more accessible and scalable. This work contributes to the growing field of ecological AI and highlights the potential for sound AI to play a crucial role in conservation and biodiversity monitoring, offering new insights into the dynamics of natural ecosystems.


Here’s a brief transition to introduce the next project:

Building on the foundations of sound-based classification developed in this project, the next phase of research focuses on generating realistic and emotionally expressive speech through Text-to-Speech (TTS) technology. This project aims to explore how synthesized voices can convey emotions, enhancing human-computer interactions by creating more lifelike and immersive auditory experiences.



\vspace*{\fill}

\section*{Acknowledgements}

I would like to express my heartfelt gratitude to the three principal investigators who have guided me over the past three years. My deepest thanks go to Marc Teyssier for his invaluable support and mentorship during my UROP year, as well as to Clément Duhart and Xiao Xiao for their guidance and assistance throughout these two years of my master’s program.

I am also deeply grateful to my classmates and friends for making these two years unforgettable. A special thanks to Hugo Devoille and Paul Even, as well as to Noé Guennoun, Nicolas Leboucher, Yohann Cossez, Marc-Adrien, and many others who have been part of this journey.

Lastly, I would like to thank my professors and researchers, including Yliess, Grégor, Thomas Juldo, Paul-Peter, and Madalina, for their unwavering support and the time they dedicated to teaching us. Their efforts have equipped us with the comprehensive knowledge and skills in engineering and research, preparing us to face future challenges with confidence.
\vspace*{\fill}





%----------------------------------------------------------------------------------------

\backmatter % Denotes the end of the main document content
\setchapterstyle{plain} % Output plain chapters from this point onwards

%----------------------------------------------------------------------------------------
%	BIBLIOGRAPHY
%----------------------------------------------------------------------------------------

% The bibliography needs to be compiled with biber using your LaTeX editor, or on the command line with 'biber main' from the template directory

\defbibnote{bibnote}{Here are the references in citation order.\par\bigskip} % Prepend this text to the bibliography
\printbibliography[heading=bibintoc, title=References, prenote=bibnote] % Add the bibliography heading to the ToC, set the title of the bibliography and output the bibliography note

%----------------------------------------------------------------------------------------
%	NOMENCLATURE
%----------------------------------------------------------------------------------------

% The nomenclature needs to be compiled on the command line with 'makeindex main.nlo -s nomencl.ist -o main.nls' from the template directory

% \nomenclature{$c$}{Speed of light in a vacuum inertial frame}
% \nomenclature{$h$}{Planck constant}

% \renewcommand{\nomname}{Notation} % Rename the default 'Nomenclature'
% \renewcommand{\nompreamble}{The next list describes several symbols that will be later used within the body of the document.} % Prepend this text to the nomenclature

% \printnomenclature % Output the nomenclature

%----------------------------------------------------------------------------------------
%	GREEK ALPHABET
% 	Originally from https://gitlab.com/jim.hefferon/linear-algebra
%----------------------------------------------------------------------------------------

% \vspace{1cm}

% {\usekomafont{chapter}Greek Letters with Pronounciation} \\[2ex]
% \begin{center}
% 	\newcommand{\pronounced}[1]{\hspace*{.2em}\small\textit{#1}}
% 	\begin{tabular}{l l @{\hspace*{3em}} l l}
% 		\toprule
% 		Character & Name & Character & Name \\ 
% 		\midrule
% 		$\alpha$ & alpha \pronounced{AL-fuh} & $\nu$ & nu \pronounced{NEW} \\
% 		$\beta$ & beta \pronounced{BAY-tuh} & $\xi$, $\Xi$ & xi \pronounced{KSIGH} \\ 
% 		$\gamma$, $\Gamma$ & gamma \pronounced{GAM-muh} & o & omicron \pronounced{OM-uh-CRON} \\
% 		$\delta$, $\Delta$ & delta \pronounced{DEL-tuh} & $\pi$, $\Pi$ & pi \pronounced{PIE} \\
% 		$\epsilon$ & epsilon \pronounced{EP-suh-lon} & $\rho$ & rho \pronounced{ROW} \\
% 		$\zeta$ & zeta \pronounced{ZAY-tuh} & $\sigma$, $\Sigma$ & sigma \pronounced{SIG-muh} \\
% 		$\eta$ & eta \pronounced{AY-tuh} & $\tau$ & tau \pronounced{TOW (as in cow)} \\
% 		$\theta$, $\Theta$ & theta \pronounced{THAY-tuh} & $\upsilon$, $\Upsilon$ & upsilon \pronounced{OOP-suh-LON} \\
% 		$\iota$ & iota \pronounced{eye-OH-tuh} & $\phi$, $\Phi$ & phi \pronounced{FEE, or FI (as in hi)} \\
% 		$\kappa$ & kappa \pronounced{KAP-uh} & $\chi$ & chi \pronounced{KI (as in hi)} \\
% 		$\lambda$, $\Lambda$ & lambda \pronounced{LAM-duh} & $\psi$, $\Psi$ & psi \pronounced{SIGH, or PSIGH} \\
% 		$\mu$ & mu \pronounced{MEW} & $\omega$, $\Omega$ & omega \pronounced{oh-MAY-guh} \\
% 		\bottomrule
% 	\end{tabular} \\[1.5ex]
% 	Capitals shown are the ones that differ from Roman capitals.
% \end{center}

%----------------------------------------------------------------------------------------
%	GLOSSARY
%----------------------------------------------------------------------------------------

% The glossary needs to be compiled on the command line with 'makeglossaries main' from the template directory

% \newglossaryentry{computer}{
% 	name=computer,
% 	description={is a programmable machine that receives input, stores and manipulates data, and provides output in a useful format}
% }

% Glossary entries (used in text with e.g. \acrfull{fpsLabel} or \acrshort{fpsLabel})
% \newacronym[longplural={Frames per Second}]{fpsLabel}{FPS}{Frame per Second}
% \newacronym[longplural={Tables of Contents}]{tocLabel}{TOC}{Table of Contents}

\setglossarystyle{listgroup} % Set the style of the glossary (see https://en.wikibooks.org/wiki/LaTeX/Glossary for a reference)
\printglossary[title=Special Terms, toctitle=List of Terms] % Output the glossary, 'title' is the chapter heading for the glossary, toctitle is the table of contents heading

%----------------------------------------------------------------------------------------
%	INDEX
%----------------------------------------------------------------------------------------

% The index needs to be compiled on the command line with 'makeindex main' from the template directory

\printindex % Output the index

%----------------------------------------------------------------------------------------
%	BACK COVER
%----------------------------------------------------------------------------------------

% If you have a PDF/image file that you want to use as a back cover, uncomment the following lines

%\clearpage
%\thispagestyle{empty}
%\null%
%\clearpage
%\includepdf{cover-back.pdf}

%----------------------------------------------------------------------------------------

\end{document}
