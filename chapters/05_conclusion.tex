\section{Conclusion}

In conclusion, this thesis explored the innovative potential of using plants as bio-sensors within a system that integrates hardware and software components to capture human-plant interaction and transform it into auditory experiences. Through the development of a standalone electronic system, including sensors, data processing capabilities, and sonification features, this research presents a brand new intersection of human-computer interaction, biology, and sound design.

The Internet of Plants (IoP) framework proposed here introduces a new way to think about plant-environment interaction, increasing human awareness of plants' responses to touch. The real-time sonification enabled by the IoP allows for immersive, interactive art installations and potential applications in fields such as agriculture and environmental monitoring. Despite the promising results, several limitations were identified, particularly regarding sensor accuracy, sound quality, and system scalability, which will require further investigation and improvement.

Future work may focus on overcoming these limitations by integrating more sophisticated sensors and audio technologies, while exploring the potential for broader IoT applications. Additionally, expanding the distributed system concept could enable more complex networks of interactive plants, offering new possibilities for environmental data collection, art, and human-plant symbiosis.