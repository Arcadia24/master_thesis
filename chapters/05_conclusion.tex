\section{Conclusion}

This thesis has explored the intersection of technology, biology, and sound through the development of a system that transforms plants into interactive bio-sensors. By integrating hardware and software, the project demonstrated the feasibility of capturing human-plant interaction and turning it into real-time auditory feedback. The following subsections revisit the research questions, outline the key contributions, discuss future implications, and provide final reflections on the work.

\subsection{Revisiting the research questions}

The main research question posed by this thesis was: How can sensing technologies redefine our interactions with plants? This study has demonstrated that by embedding plants with sensor technologies, it is possible to capture their responses to touch and transform them into auditory experiences. This redefines human-plant interaction, turning plants into dynamic, interactive entities. The system encourages a deeper connection with nature by allowing people to engage with plants in ways beyond the visual and tactile, creating a multi-sensory interface that transforms plants into responsive and communicative organisms.

\subsection{Contributions to the Field}

This research makes significant contributions to multiple domains:
\begin{itemize}
    \item \textbf{Human-Computer Interaction (HCI)}: By turning plants into bio-sensors, the thesis introduces a new interface for interaction that connects humans and nature through technology.
    \item \textbf{Sound Design and Sonification}: The system developed provides a novel way to transform data from plants into auditory feedback, contributing to the expanding field of sonification.
    \item \textbf{IoT and Sensor Networks}: The introduction of the Internet of Plants (IoP) framework extends the idea of interconnected sensor networks by using plants as nodes, bridging technology with the natural environment in a new way.
    \item \textbf{Artistic and Ecological Applications}: The potential for using plant-based bio-sensors in interactive art installations and environmental monitoring systems opens up interdisciplinary opportunities.
\end{itemize}
  

\subsection{Future implications}

The implications of this research extend beyond art and human-computer interaction. The Internet of Plants (IoP) framework could have applications in agriculture, where real-time monitoring of plant health could improve crop management. Additionally, the integration of more advanced sensors and audio technologies could lead to more complex systems capable of processing detailed environmental data, enhancing both artistic experiences and practical applications like ecological monitoring.


\subsection{Final thoughts}

This thesis offers a glimpse into the future of human-nature interaction, where plants serve not only as aesthetic or ecological entities but as active participants in a technological system. While the system shows great potential, challenges related to sensor accuracy, sound quality, and scalability remain. Addressing these limitations in future research could fully unlock the potential of plant-based bio-sensors, advancing both the Internet of Plants and sustainable, nature-integrated technology.