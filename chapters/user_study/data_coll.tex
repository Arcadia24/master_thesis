\section{Data collection}
To capture the nuances of participant interactions with the conceptualized musical plants, a collaborative approach was adopted, involving two researchers to provide dual perspectives. Throughout the exploration phase, both researchers meticulously took notes, documenting the diverse ways in which participants engaged with the three distinct plants. Each note explicitly specified the plant involved in the interaction, ensuring a granular understanding of the responses tied to each botanical entity.

The notes encompassed detailed descriptions of participants' actions, expressions, and verbalizations, aiming to encapsulate the richness of their experiences. The dual-observer strategy facilitated a more comprehensive and triangulated perspective, mitigating potential biases and enhancing the reliability of the recorded data. The collaborative note-taking process served as a valuable means of capturing the multifaceted nature of human-plant interaction within the experimental context, contributing to the depth and richness of the findings in the subsequent analysis.

We defined 5 possible interaction :

\begin{itemize}
    \item Grasp : user uses the whole hand to grab trunk or leaves
    \item Pinch : user uses 2 to 3 digits to grab trunk or leaves
    \item Slide : user uses his/her hand or finger to slide on the plant
    \item Pet : user uses his/her hand to cuddle the plant or to pass through the leaves
    \item Tam Tam : user taps on the plant mainly using the whole hand
\end{itemize}

% \hl{explain the class of interaction}
