\section{Introduction}


Plants represent a full ecosystem of evolution, adaptation and communication.

In the context of the Internet of Plant (IoP) project, this study aims to extract the natural interaction between people and plants.
This experiment explores the interactions the IoP device will have to detect to create a symbiotic relation between human and plants. 
The physical touch is the starting point of a sonification process.
Sonification is “the use of non-speech audio to convey information or perceptualize data” \cite{kramer2010sonification}.
Three distinct plant species—\textit{Dypsis lutescens, Pachira glabra, and Dracaena}—are employed as subjects to extract user perceptions and interactions within this framework. 

% Three plant species, Dypsis lutescens, Pachira glabra, and Dracaena, serve as conduits for inquiry, 
% prompting participants to envisage that these botanical entities create and generate sounds upon physical interaction.

The methodology engages students from the engineering school and two researchers.
The participants are asked to interact with the plants and imagine the sounds that could be generated by the plants.

The correlation between plant height and trunk interactions reveals environmental factors impacting human-plant dynamics.
Additionally, interactions are categorized based on intensity, spatial displacement, and duration.

