\section{Conclusion}

During our study on the Internet of Plant project, we've captured insights into how people might interact with plants in a future where they make music through touch.

Our three chosen plants influenced how participants engaged with them. We observed everything from gentle petting to energetic drumming on the plants.
Interestingly, we found that when the plants were higher up, participants tended to focus more on the trunk.

By grouping interactions based on factors like intensity and duration, we gained a clearer picture of how people approached these musical plants.
It turns out that certain interactions, like grasping and pinching, were more common, while others, like sliding, had their own distinct appeal.

Regarding to the results we thought about what could be done with the defined interactions.
For instance, the sound generated from the interaction could be linked to the kind of interaction.
People doing Tam Tam on the plant will expect a drum sound. Whereas, people performing a slide will expect a sound closer to a continuous organ sound.
The possibilities are endless and the only restrictions are the capabilities of the device capturing the interaction. 