\section{Introduction}

% \subsection{Background motivation}

% I am a technology engineer specializing in embedded systems and their hardware/software architecture. At the request of my principal investigator, I developed a device that uses sensor data from plants to generate sound based on human-plant interactions. Despite having no prior experience with musical systems, this project allowed me to explore how data from plant sensors could be transformed into sound, providing a way to engage with plants through auditory feedback.

% Building on this work, I am also focused on creating a network of plant-based sensors. By using the natural sensing capabilities of plants, I aim to extend the functionality of individual plants into a larger system that can capture and record interactions, forming a connected network of bio-sensors.

\subsection{Context and overview}

The development of new interaction methods and sensor technologies is rapidly evolving, driven by the increasing integration of advanced sensing devices into everyday life. This thesis explores an unconventional approach to sensor systems by exploiting the natural sensing capabilities of plants. Plants are naturally provided with thousands of sensors that respond to environmental stimuli, making them a resource for monitoring and interaction technologies.

The primary goal of this research is to understand and utilize the sensing abilities of plants, transforming them into bio-sensors. By doing so, the project aims to create networks of interconnected plants that can monitor their own environmental states and respond to human interaction. This approach could provide a sustainable alternative to traditional silicon-based sensor networks, potentially reducing the reliance on artificial materials and the environmental impact associated with them.

Additionally, this research has implications for the field of Human-Computer Interaction (HCI), offering new opportunities to explore plant-based interfaces. By incorporating plants into interaction systems, this thesis could introduce a natural interface that puts the spotlight on a more integrated relationship between humans and their surrounding environments. This exploration of bio-sensors as part of HCI opens up new paths for sustainable, living technologies that go beyond conventional electronic systems.


\subsection{Problem statement}

The main problem that this master thesis will focus on is :

\begin{center}
    \textbf{How can sensing technologies redefine our interactions with plants ?}\\
\end{center}

\subsection{Research domain}

The study of human-plant interaction covers a broad spectrum of research areas. This master's thesis focuses on the Human-Computer Interaction (HCI) field, which examines the interfaces between people and computers. HCI sits "at the intersection between psychology and social sciences, on the one hand, and computer science and technology, on the other," \cite{carrollHUMANCOMPUTERINTERACTIONPsychology} providing insights into how humans interact with technology.

In this thesis, the concept of interaction is extended to plants and nature, aiming to enhance plant capabilities and explore how plants can be used as interactive elements. By transforming plants into living sensors, the research also intersects with the fields of instrumentation engineering and electronics. These fields focus on developing new ways of capturing data to design functional sensors. Understanding plants as bio-living sensors requires the application of specific sensing techniques to access their natural abilities.

Additionally, this work involves the sensor development field, where plants are converted into functional sensors. This transformation represents an innovative approach to expanding the possibilities of plant-based sensing technologies.


\subsection{Contributions}

This thesis makes several contributions to the field of Human-Computer Interaction and sensor technologies:

\begin{enumerate}
    \item \textbf{Development of Plant-based Bio-sensors and Sound Interaction System}: This thesis introduces a novel system that transforms plants into living bio-sensors, utilizing their natural sensing capabilities to capture human-plant interactions. The data collected from plant sensors is used to generate real-time auditory feedback, creating an innovative, sound-based interaction mechanism. This approach opens up new avenues for exploring plant interactions in both technological and artistic contexts.
    \item \textbf{Creation of a Network of Plant Sensors for Distributed Interaction}: In addition to the single-plant sensor system, this research extends the concept by developing a network of interconnected plant-based sensors. This network allows for distributed sensing and interaction across multiple plants, enabling more complex data collection and processing. The system demonstrates potential applications in human-computer interaction, environmental monitoring, and artistic installations, highlighting its multidisciplinary impact.
\end{enumerate}
