\section{Introduction}

\subsection{Background motivation}

I am an creative technology engineer that is passionate about embedded systems and 
their hardware/software architecture.
Pushed by my principal investigator and motivated by challenges, I wanted to 
explore the intersection of biology and electronics. 
I aim to transform plants into bio-sensors, using their natural sensing capabilities to 
capture the human-plant interaction. Extending the capacities of a single plant, I want to create a network of 
plant-based sensors. 

I am also interested in the use of sensor data. With no particular appetence for musical creation, 
my principal investigator challenged me onto create a device that can use the data from the plant 
and generate sound based on interaction. The musical generation allows the plant to be listened to and to \hl{care
about it.}

\subsection{Context and overview}

This research is in line with the new means of interaction and new sensors that surround us every day.
This master's thesis seeks to use the natural capacities of plants, which are made up of thousands of sensors, and to understand them.
This could make it possible to create plant networks and monitor the state of our green partners. 
At the same time, it could reduce the amount of silicon needed to deploy a sensor field.
It could also open up new possibilities in the field of Human Computer Interface research by adding a new interface.


\subsection{Problematic}

The main problematic that this master thesis will focus on is :

\begin{center}
    \textbf{How can sensing technologies redefine our interactions with plants ?}\\
\end{center}

\subsection{Research domain}

Research domains on the human-plant interaction are wide. The HCI\footnote{Human Computer Interaction} field is focused
during this master thesis. The Human Computer Interaction field focuses on the interfaces between people and computers.
This field is at the intersection "between psychology and social sciences, on the one hand, and computer science and technology,
on the other" \cite{carrollHUMANCOMPUTERINTERACTIONPsychology}. This master thesis aims to work onto the interaction
we have with plants and nature and to enhance plant capabilities.

The plant is transformed is used as a living sensor and thus the project is reaching the instrumentation engineering 
and electronic field. This field aims to think and create new way of capturing data to make sensors.
A bio-living sensor such as the plant needs to be understood using sensing techniques.

The sensor making and creation field is also focused. In this master thesis plants are transformed into sensor.
This transformation...


\subsection{Contributions}